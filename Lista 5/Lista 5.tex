\documentclass[a4paper, 12pt]{article}

\usepackage[top=2cm, bottom=2cm, left=2.5cm, right=2.5cm]{geometry}
\usepackage[utf8]{inputenc}
\usepackage{amsmath, amsfonts, amssymb}
\usepackage{graphicx} % inserir figuras - \includegraphics[scale=•]{•}
\usepackage{float} % ignorar regras de tipografia e inserir figura aonde queremos.
\usepackage[brazil]{babel} % Trocar Figure para Figura.
\usepackage{indentfirst}
\pagestyle{empty}


\begin{document}
\begin{figure}[htb]
	\includegraphics[scale=0.9]{UnB_CiC_Logo.jpg}
\end{figure}
\noindent\rule{\textwidth}{0.4pt}
\begin{center}
	\textbf{{\Large Introdução à Ciência da Computação - 113913}} \newline \newline
	\textbf{{\large Lista de Exercícios 5} \\
	\vspace{9pt}
	{\large Strings}} \\
	\noindent\rule{\textwidth}{0.4pt}
	\newline
\end{center}

\textbf{{\large Observações:}}
\begin{itemize}
	\item As listas de exercícios serão corrigidas por um \textbf{corretor automático}, portanto é necessário que as entradas e saídas do seu programa estejam conforme o padrão especificado em cada questão (exemplo de entrada e saída). Por exemplo, não use mensagens escritas durante o desenvolvimento do seu código como ``Informe a primeira entrada''. Estas mensagens não são tratadas pelo corretor, portanto a correção irá resultar em resposta errada, mesmo que seu código esteja correto.
	\item As questões estão em \textbf{ordem de dificuldade}. Cada lista possui 7 exercícios, sendo 1 questão fácil, 3 ou 4 médias e 2 ou 3 difíceis.
	\item Assim como as listas, as provas devem ser feitas na versão Python 3 ou superior.
	\item Leia com atenção e faça \textbf{exatamente} o que está sendo pedido.
\end{itemize}
\newpage % Questão A 
\begin{center}
\textbf{{\Large Questão A - Ataque em Roma}}
\end{center}
\vspace{5pt}
A equipe de inteligência Romana tem obtido sucesso interceptando as mensagens do Império Otomano, mas elas são muito suspeitas, pois não faziam sentido algum. Até que Aristolfo descobriu o segredo adversário: a real mensagem se esconde no terceiro caractere de cada uma das palavras. Por exemplo:
\begin{center} \textit{An\textbf{d}ré tr\textbf{e}tou ca\textbf{b}um br\textbf{o}cado tr\textbf{a}ns pu\textbf{s}}\end{center}
Vira:
\begin{center} \textit{\textbf{deboas}}\end{center}
Até então, este trabalho estava sendo feito à mão. Mas como você é muito
inteligente, Aristolfo pediu que escrevesse um programa que resolvesse isso
para ele.
\newline \newline
\textbf{{\large Entrada}} \newline
A entrada consiste em uma frase criptografada usando o método descrito acima, contendo apenas palavras de três letras ou mais.
\newline \newline
\textbf{{\large Saída}} \newline
Seu programa deve escrever uma única linha na saída, contendo a real mensagem do Império Otomano.
\newline
\begin{table}[H]
	\centering
	\begin{tabular}{|l|l|}
\hline
\textbf{Exemplo de Entrada}                                                                         & \textbf{Exemplo de Saída} \\ \hline
\begin{tabular}[c]{@{}l@{}}André tretou cabum brocado trans\\ pus\end{tabular}                      & deboas                    \\ \hline
\begin{tabular}[c]{@{}l@{}}Esparta prestigia derrotas trincadas\\ legais problemáticas\end{tabular} & perigo                    \\ \hline
\end{tabular}
	\caption{Questão A}
	\label{tabela1}
\end{table}

\newpage % Questão B
\begin{center}
\textbf{{\Large Questão B - Brincadeira de Criança}}
\end{center}
\vspace{5pt}
André adora brincar com seus amiguinhos usando a língua do ‘p’, mas nem
sempre eles conseguem reproduzir com tanta maestria as frases trocando
todas as consoantes por ‘p’. \newline \newline
Para ajudar André a não perder mais tempo explicando para os seus
coleguinhas como é a pronúncia correta de tão sublime linguagem, você
decidiu escrever um programa que processa uma frase e retorna ela
traduzida.
\newline \newline
\textbf{{\large Entrada}} \newline
A entrada consiste em uma única linha contendo a frase a ser traduzida. A
frase pode ser de um tamanho arbitrário e não conterá símbolos especiais.
\newline \newline
\textbf{{\large Saída}} \newline
Seu programa deve escrever uma única linha na saída, contendo a frase com
todas as suas consoantes substituídas pela letra ‘p’.
\newline
\begin{table}[H]
	\centering
	\begin{tabular}{|l|l|}
\hline
\textbf{Exemplo de Entrada}                                                                         & \textbf{Exemplo de Saída}                                                                           \\ \hline
\begin{tabular}[c]{@{}l@{}}André tretou cabum brocado trans\\ pus\end{tabular}                      & \begin{tabular}[c]{@{}l@{}}Apppé ppepou papup ppopapo \\ ppapp pup\end{tabular}                     \\ \hline
\begin{tabular}[c]{@{}l@{}}Esparta prestigia derrotas trincadas\\ legais problemáticas\end{tabular} & \begin{tabular}[c]{@{}l@{}}Eppappa ppeppipia peppopap\\ ppippapap pegaip ppoppepápipap\end{tabular} \\ \hline
\end{tabular}
	\caption{Questão B}
	\label{tabela2}
\end{table}

\newpage % Questão C
\begin{center}
\textbf{{\Large Questão C - Cruzeiro}}
\end{center}
\vspace{5pt}
Andreia e seu amigo, Brocado, estão jogando um jogo que eles mesmo
inventaram, o Cruzeiro. Neste jogo, cada um deles escreve uma palavra
escondido num papel, e depois as comparam. Ganha quem tiver escrito a
palavra que, em ordem alfabética (\textit{lexicográfica}) vem ``depois''.
\newline \newline
Por exemplo, se Andreia joga ``ornitorrinco'', mas Brocado joga ``ornitopata'',
vence Andreia, pois a letra ‘r’ vem depois do ‘p’.
\newline \newline
Porém, depois de um tempo, as duas crianças cansam de ficar tentando
``calcular'' qual das duas palavras vence. Você, sendo um inteligente
programador, se dispõe a ajudá-las, escrevendo um programa que responde quem é o vencedor. 
\newline \newline
\textbf{{\large Entrada}} \newline
A entrada consistem em duas strings diferentes \textbf{A} e \textbf{B} separadas por espaço,
relativas às palavras submetidas por Andreia e Brocado, respectivamente.
\newline \newline
\textbf{{\large Saída}} \newline
Seu programa deve escrever uma única linha, contendo apenas um caractere
‘A’ ou ‘B’, com a inicial do nome do vencedor.
\newline
\begin{table}[H]
	\centering
	\begin{tabular}{|l|l|}
\hline
\textbf{Exemplo de Entrada} & \textbf{Exemplo de Saída} \\ \hline
sino cabeça                 & A                         \\ \hline
lontra Pele                 & B                         \\ \hline
\end{tabular}
	\caption{Questão C}
	\label{tabela3}
\end{table}

\newpage % Questão D
\begin{center}
\textbf{{\Large Questão D - Discussão Interminável}}
\end{center}
\vspace{5pt}
Panda e Urso estão discutindo para saber qual letra é a mais recorrente em
textos clássicos, ‘p’ ou ‘u’. \newline \newline
Sabendo que você é um experiente programador, Panda e Urso vêm pedir
seu auxílio para dar um fim a esta disputa.
\newline \newline
\textbf{{\large Entrada}} \newline 
A primeira linha da entrada contém apenas um inteiro \textbf{\textit{N}}, indicando o número
de linhas do texto. Cada uma das seguintes linhas (onde o fim é determinado pelo ponto final) contém uma linha do texto,
que não contém caracteres acentuados.
\newline \newline
\textbf{{\large Saída}} \newline
Seu programa deve escrever na saída dois inteiros \textbf{P} e \textbf{U}, correspondentes
ao número de ocorrências das letras ‘p’ e ‘u’, respectivamente.
\newline
\begin{table}[H]
	\centering
	\begin{tabular}{|l|l|}
\hline
\textbf{Exemplo de Entrada}                                                                                                                                                                                                                                  & \textbf{Exemplo de Saída} \\ \hline
\begin{tabular}[c]{@{}l@{}}3\\ Lorem ipsum dolor sit amet, \\ consectetur adipiscing elit.\\ Nam vestibulum fringilla lacus sed\\ tincidunt.\\ Pellentesque pellentesque feugiat \\ lobortis.\end{tabular}                                                   & 4 9                       \\ \hline
\begin{tabular}[c]{@{}l@{}}2\\ Era uma vez um abacate, que estava \\ indo dormir na casa da namorada, \\ mas la tinha um panda de pelucia.\\ Entao, o abacate nao estava \\ conseguindo dormir porque achou\\ que o panda estava encarando ele.\end{tabular} & 4 9                       \\ \hline
\end{tabular}
	\caption{Questão D}
	\label{tabela4}
\end{table}

\newpage % Questão E
\begin{center}
\textbf{{\Large Questão E - Erro de Português}}
\end{center}
\vspace{5pt}
Seu sobrinho Joberto está aprendendo a escrever e está tendo dificuldade de
lembrar que deve usar letra maiúscula no início da frase. \newline \newline
Como um bom tio que é, e um exímio programador, você decide que a melhor
abordagem é escrever um programa que faça as correções automaticamente.
\newline \newline
\textbf{{\large Entrada}} \newline
A entrada consiste de uma única linha de texto, contendo uma ou múltiplas
frases, separadas por ponto.
\newline \newline
\textbf{{\large Saída}} \newline
Seu programa deve capitalizar a primeira letra da string, bem como a primeira
letra de cada frase separada por ponto.
\newline
\begin{table}[H]
	\centering
	\begin{tabular}{|l|l|}
\hline
\textbf{Exemplo de Entrada}                                                                                                                           & \textbf{Exemplo de Saída}                                                                                                                             \\ \hline
\begin{tabular}[c]{@{}l@{}}quando eu fui ganhar um abraço, eu\\ não esperava. que ele fosse ser tão\\ sincero.\end{tabular}                           & \begin{tabular}[c]{@{}l@{}}Quando eu fui ganhar um abraço, eu\\ não esperava. Que ele fosse ser tão sincero.\end{tabular}                             \\ \hline
\begin{tabular}[c]{@{}l@{}}abacates podem ser verdes. mas\\ também existem abacates marrons.\\ E não se esqueça dos que não são\\ roxos.\end{tabular} & \begin{tabular}[c]{@{}l@{}}Abacates podem ser verdes. Mas\\ também existem abacates marrons.\\ E não se esqueça dos que não são\\ roxos.\end{tabular} \\ \hline
\end{tabular}
	\caption{Questão E}
	\label{tabela5}
\end{table}

\newpage % Questão F
\begin{center}
\textbf{{\Large Questão F - Freguês}}
\end{center}
\vspace{5pt}
John é garçom de um restaurante cujo prato de maior sucesso é a sopa de
letrinhas. Para passar o tempo, ele fica procurando as letras do seu nome na
sopa que entrega para os clientes. \newline \newline
Enquanto fazia isso, percebeu que, dentre as sopas que os clientes deixavam
para trás ao sair, existia um padrão nas letrinhas restantes.
As letras do seu nome sempre eram rejeitadas! \newline \newline
Frustrado e levemente assustado com a descoberta, John pede para que
você escreva um programa para se certificar que não está ficando maluco.
\newline \newline
\textbf{{\large Entrada}} \newline
A primeira linha da entrada consiste numa string contendo todas as letras da
sopa de letras entregue. Note que poderá haver repetição. \newline	
A segunda linha, por sua vez, consiste numa string contendo todas as letras
da sopa abandonada pelo cliente.
\newline \newline
\textbf{{\large Saída}} \newline
Seu programa deve imprimir na saída dois inteiros \textbf{\textit{I}}, \textbf{\textit{O}}, correspondendo ao número de letras \textbf{J}, \textbf{O}, \textbf{H} ou \textbf{N} que foram comidas pelo cliente e que foram deixadas para trás, respectivamente.
\newline
\begin{table}[H]
	\centering
	\begin{tabular}{|l|l|}
\hline
\textbf{Exemplo de Entrada}                                                   & \textbf{Exemplo de Saída} \\ \hline
\begin{tabular}[c]{@{}l@{}}ABCDEFGHIJKLMNOPQRSTUVWX\\ YZ\\ ABCDE\end{tabular} & 4 0                       \\ \hline
\begin{tabular}[c]{@{}l@{}}ABCDEFGHIJKLMNOPQRSTUVWX\\ YZ|\\ JOHN\end{tabular} & 0 4                       \\ \hline
\end{tabular}
	\caption{Questão F}
	\label{tabela6}
\end{table}

\newpage % Questão G
\begin{center}
\textbf{{\Large Questão G - Gradiente}}
\end{center}
\vspace{5pt}
Raphael está desenvolvendo um analisador de gradiente léxico para o seu mais novo experimento com Inteligência Artifical e processamento de texto em linguagem natural. Para isso, ele precisa saber, com precisão, se os caracteres de uma string são maiúsculos, minúsculos, espaços em branco ou dígitos numéricos. \newline \newline
Sabendo que você é um exímio programador, Raphael pediu sua ajuda nesta tarefa. \newline \newline
\textbf{{\large Entrada}} \newline
A entrada consiste em uma única linha de texto, podendo conter letras maiúsculas, minúsculas, espaços em branco e dígitos numéricos, mas não caracteres especiais.
\newline \newline
\textbf{{\large Saída}} \newline
Seu programa deve gerar uma única linha de texto, mapeando os caracteres
da entrada para \textbf{U}, para letras maiúsculas, \textbf{L}, para letras minúsculas, \textbf{W}, para
espaços em branco e \textbf{D} para dígitos numéricos.
\newline
\begin{table}[H]
	\centering
	\begin{tabular}{|l|l|}
\hline
\textbf{Exemplo de Entrada}   & \textbf{Exemplo de Saída}                                                \\ \hline
iwHeJokl 2387E23ei jDiae      & LLULULLLWDDDDUDDLLWLULLL                                                 \\ \hline
iw iWw potato JJJJklo 123 456 & \begin{tabular}[c]{@{}l@{}}LLWLULWLLLLLLWUUUULLLWDDD\\ WDDD\end{tabular} \\ \hline
\end{tabular}
	\caption{Questão G}
	\label{tabela7}
\end{table}
\end{document}