\documentclass[a4paper, 12pt]{article}

\usepackage[top=2cm, bottom=2cm, left=2.5cm, right=2.5cm]{geometry}
\usepackage[utf8]{inputenc}
\usepackage{amsmath, amsfonts, amssymb}
\usepackage{graphicx} % inserir figuras - \includegraphics[scale=•]{•}
\usepackage{float} % ignorar regras de tipografia e inserir figura aonde queremos.
\usepackage[brazil]{babel} % Trocar Figure para Figura.
\usepackage{indentfirst}
\pagestyle{empty}


\begin{document}
\begin{figure}[H]
	\includegraphics[scale=0.9]{UnB_CiC_Logo.jpg}
\end{figure}
\noindent\rule{\textwidth}{0.4pt}
\begin{center}
	\textbf{{\Large Introdução à Ciência da Computação - 113913}} \newline \newline
	\textbf{{\large Lista de Revisão} \\
	\vspace{9pt}
	{\large Prova 1}} \\
	\noindent\rule{\textwidth}{0.4pt}
	\newline
\end{center}

\textbf{{\large Observações:}}
\begin{itemize}
	\item As provas também serão corrigidas por um \textbf{corretor automático}, portanto é necessário que as entradas e saídas do seu programa estejam conforme o padrão especificado em cada questão (exemplo de entrada e saída). Por exemplo, não use mensagens escritas durante o desenvolvimento do seu código como ``Informe a primeira entradas''. Estas mensagens não são tratadas pelo corretor, portanto a correção irá resultar em resposta errada, mesmo que seu código esteja correto.
	\item Serão testadas várias entradas além das que foram dadas como exemplo, assim como as listas.
	\item Assim como as listas, as provas devem ser feitas na versão Python 3 ou superior.
	\item Leia com atenção e faça \textbf{exatamente} o que está sendo pedido.
\end{itemize}

\newpage % Questão A 
\begin{center}
\textbf{{\Large Questão A - Máximo Divisor Comum}}
\end{center}
\vspace{5pt}
O máximo divisor comum entre dois ou mais números inteiros é o maior número inteiro que é fator de tais números. Por exemplo, os divisores comuns de 12 e 18 são 1, 2, 3 e 6, logo \textbf{\textit{mdc(12,18) = 6}}. Dizemos que dois números inteiros a e b são primos entre si, se e somente se \textbf{\textit{mdc(a,b) = 1}}. Faça um programa que leia uma sequência de duplas de inteiros do teclado, \textbf{A} e \textbf{B}. A quantidade de duplas da sequência é desconhecida, mas ela termina quando \textbf{A} ou \textbf{B} for menor ou igual a zero. A dupla que contém \textbf{A} ou \textbf{B} menor ou igual a zero não faz parte da sequência, devendo ser desconsiderada.
\newline \newline
\textbf{{\large Entrada}} \newline
A entrada será a sequência de duplas de inteiros, cada linha de entrada contém dois inteiros \textbf{A} e \textbf{B}, separados por espaço. Considere que a sequência contém pelo menos uma dupla.
\newline \newline
\textbf{{\large Saída}} \newline
Para cada \textbf{A} e \textbf{B} lidos que fazem parte da sequência, calcule e imprima na tela \textbf{\textit{mdc(A,B)}}. Ao final imprima a média de todos os máximos divisores comuns calculados com duas casas decimais após a vírgula.
\newline
\begin{table}[H]
\centering
\begin{tabular}{|l|l|}
\hline
\textbf{Exemplo de Entrada}                                       & \textbf{Exemplo de Saída}                             \\ \hline
\begin{tabular}[c]{@{}l@{}}8 12\\ 7 9\\ 397 311\\ -0 4\end{tabular}   & \begin{tabular}[c]{@{}l@{}}4\\ 1\\ 1 \\ 2.00\end{tabular} \\ \hline
\begin{tabular}[c]{@{}l@{}}8 13\\ 8 14\\ 4 0\end{tabular}  & \begin{tabular}[c]{@{}l@{}}1\\ 2\\ 1.50\end{tabular} \\ \hline
\begin{tabular}[c]{@{}l@{}}16 120 \\ -1 -1\end{tabular} & \begin{tabular}[c]{@{}l@{}}8\\ 8.00 \end{tabular} \\ \hline
\end{tabular}
\caption{Questão A}
\end{table}

\newpage % Questão A 
\begin{center}
\textbf{{\Large Questão B - Mínimo Múltiplo Comum}}
\end{center}
\vspace{5pt}
O mínimo múltiplo comum (mmc) de dois inteiros a e b é o menor inteiro positivo que é múltiplo simultaneamente de a e de b. Se não existir tal inteiro positivo, por exemplo, se a = 0 ou b = 0, então \textit{\textbf{mmc(a,b)}} é zero por definição. Sabemos que $a\cdot b = mmc(a,b) \cdot mdc(a,b)$.
\newline \newline
\textbf{{\large Entrada}} \newline
A entrada contém apenas valores inteiros, sendo \textbf{\textit{N}} $>$ \textbf{\textit{0}} e \textbf{\textit{A,B}} $\geq$ \textbf{\textit{0}}. Na primeira linha será lido o valor \textbf{N} e nas próximas \textbf{N} linhas serão lidos os valores \textbf{A} e \textbf{B}, separados por espaço.
\newline \newline
\textbf{{\large Saída}} \newline
Para cada valor \textbf{\textit{A}} e \textbf{\textit{B}} lidos, calcule e imprima seu mmc. Ao final, imprima a média dos mínimos múltiplos comuns (com duas casas decimais após a vírgula) dos mmcs calculados.
\newline
\begin{table}[H]
\centering
\begin{tabular}{|l|l|}
\hline
\textbf{Exemplo de Entrada}                                       & \textbf{Exemplo de Saída}                             \\ \hline
\begin{tabular}[c]{@{}l@{}}2 \\ 4 8\\ 3 5\end{tabular}   & \begin{tabular}[c]{@{}l@{}}8\\ 15\\ 11.50 \end{tabular} \\ \hline
\begin{tabular}[c]{@{}l@{}}2 \\ 0 5\\ 5 0\end{tabular}  & \begin{tabular}[c]{@{}l@{}}0\\ 0\\ 0.00\end{tabular} \\ \hline
\begin{tabular}[c]{@{}l@{}}3 \\ 12 8 \\ 3 4\\ 0 2\end{tabular} & \begin{tabular}[c]{@{}l@{}}24\\ 12 \\ 0 \\ 12.00\end{tabular} \\ \hline
\begin{tabular}[c]{@{}l@{}}1 \\ 8 24\end{tabular} & \begin{tabular}[c]{@{}l@{}}24\\ 24.00\end{tabular} \\ \hline
\begin{tabular}[c]{@{}l@{}}2 \\ 4 2\\8 10\end{tabular} & \begin{tabular}[c]{@{}l@{}}4\\ 40 \\ 22.00\end{tabular} \\ \hline
\end{tabular}
\caption{Questão B}
\end{table}

\newpage % Questão A 
\begin{center}
\textbf{{\Large Questão C - Função Sigma e Tal}}
\end{center}
\vspace{5pt}
A função sigma denotada por $\sigma(n)$ é a função que soma os divisores distintos de \textbf{\textit{n}}, \textbf{incluindo 1 e n}. A função tal denotada por $\tau(n)$ é a função que retorna
a quantidade de divisores distintos de \textbf{\textit{n}}, \textbf{incluindo 1 e n}. 
\newline \newline
\textbf{{\large Entrada}} \newline
A entrada consiste de um inteiro \textbf{n}, onde $\textrm{\textbf{n}} \geq \textrm{\textbf{1}}$.
\newline \newline
\textbf{{\large Saída}} \newline
A saída será composta de 3 linhas: a primeira linha conterá todos os divisores de \textbf{\textit{n}} separados por espaço, em uma única linha, conforme exemplo abaixo. \textbf{Não deve haver espaços em branco após o último valor da linha}. A segunda linha será o valor $\sigma(n)$, e a terceira $\tau(n)$.
\newline \newline
\textbf{{\large Nota}} \newline
No primeiro exemplo, o número 4 tem três divisores: 1, 2 e 4. $\sigma(4) = 1 + 2 + 4 = 7$ e $\tau(4) = 3$.
\newline
\begin{table}[H]
\centering
\begin{tabular}{|l|l|}
\hline
\textbf{Exemplo de Entrada} & \textbf{Exemplo de Saída}                                                 \\ \hline
4                           & \begin{tabular}[c]{@{}l@{}}1 2 4\\ 7\\ 3\end{tabular}                     \\ \hline
5                           & \begin{tabular}[c]{@{}l@{}}1 5\\ 6\\ 2\end{tabular}                       \\ \hline
12                          & \begin{tabular}[c]{@{}l@{}}1 2 3 4 6 12\\ 28\\ 6\end{tabular}             \\ \hline
100                         & \begin{tabular}[c]{@{}l@{}}1 2 4 5 10 20 25 50 100\\ 217\\ 9\end{tabular} \\ \hline
50                          & \begin{tabular}[c]{@{}l@{}}1 2 5 10 25 50\\ 93\\ 6\end{tabular}           \\ \hline
\end{tabular}
\caption{Questão C}
\end{table}

\newpage % Questão A 
\begin{center}
\textbf{{\Large Questão D - The Winter is Coming}}
\end{center}
\vspace{5pt}
Os Starks sempre avisaram: ``The Winter is Coming'' e o inverno finalmente chegou em Westeros. O Rei do Norte, Jon Snow, decidiu igualar o ouro entre todas as casas do Norte, dando ouro para algumas. Para isso, ele pediu para você, o Mestre da Moeda, considerar o ouro (em kg) que cada uma possui e calcular o custo mínimo do presente do rei, sabendo que: no Norte existem \textbf{\textit{n}} casas, o ouro que cada uma possui é estimado em um inteiro $a_i$ e que o rei apenas dará ouro, não tirará de ninguém.
\newline \newline
\textbf{{\large Entrada}} \newline
A primeira linha contém um inteiro \textbf{\textit{n}}
 $(\textrm{\textbf{1}}\, \leq \textrm{\textbf{n}}\, \leq \textrm{\textbf{100}})$ - o número de casas no Norte.
As próximas \textbf{\textit{n}} linhas contém os inteiros $a_1$, $a_2$, $a_3$, \dots, $a_n$, onde 
$a_i \geq 0$ corresponde ao ouro, em kg, que cada casa possui. Considere que o primeiro inteiro $a_i$ sempre será o ouro correspondente da casa que \textbf{possui mais ouro}.
\newline \newline
\textbf{{\large Saída}} \newline
Um único inteiro que corresponde a quantidade mínima de ouro (em kg) que Winterfell irá gastar para que todas as casas tenham a mesma quantidade de ouro.
\newline \newline
\textbf{{\large Nota}} \newline
No primeiro exemplo se adicionarmos para a segunda casa 4 kg, para a terceira 3 e para a quarta 2, então todas elas terão 4 kg. \newline
No quarto exemplo não é possível dar nada para ninguém, porque todas as casas possuem 12 kg.
\begin{table}[H]
\centering
\begin{tabular}{|l|l|}
\hline
\textbf{Exemplo de Entrada}                               & \textbf{Exemplo de Saída} \\ \hline
\begin{tabular}[c]{@{}l@{}}4\\ 4\\ 0\\ 1\\ 2\end{tabular} & 9                         \\ \hline
\begin{tabular}[c]{@{}l@{}}3\\ 1\\ 1\\ 0\end{tabular}     & 1                         \\ \hline
\begin{tabular}[c]{@{}l@{}}2\\ 3\\ 1\end{tabular}         & 2                         \\ \hline
\begin{tabular}[c]{@{}l@{}}1\\ 12\end{tabular}            & 0                         \\ \hline
\end{tabular}
\caption{Questão D}
\end{table}

\newpage % Questão A 
\begin{center}
\textbf{{\Large Questão E - Fibonacci}}
\end{center}
\vspace{5pt}
Leia uma sequência de inteiros positivos do teclado, um por linha. A sequência termina quando for lido um inteiro menor ou igual a 0 (que não fará parte da sequência de números lidos). Para cada número 
\textbf{k} $>$ \textbf{0} lido, calcule o \textbf{k-ésimo}\,($F_k$) elemento da sequência de Fibonacci, conforme definição dada abaixo:
$$F_n =
		\begin{cases}
			1;\, n = 1\ \textrm{ou}\ n = 2 \\
			F_{n-1} + F_{n-2};\, n > 2 \\
		\end{cases}
$$
\newline \newline
\textbf{{\large Entrada}} \newline
Cada linha de entrada conterá um inteiro \textbf{\textit{k}}, quando a linha conter 
$k\, \leq \, 0$ o programa deve parar. Considere que pelo menos um \textbf{k} $>$ \textbf{0} será lido.
\newline \newline
\textbf{{\large Saída}} \newline
Considerando o valor de $F_k$:
\begin{itemize}
\item Caso $F_k$ seja par e k seja par, imprima a soma dos dois.
\item Caso $F_k$ seja par e k seja ímpar, imprima a diferença de $F_k$ com k.
\item Caso $F_k$ seja ímpar e k par, imprima a multiplicação.
\item Caso $F_k$ seja ímpar e k seja ímpar, imprima a divisão inteira de $F_k$ por k.
\end{itemize}
Ao final, informa a média aritmética dos números lidos da sequência com duas casas decimais e o maior $F_k$ calculado, conforme exemplo abaixo.
\newline
\begin{table}[H]
\centering
\begin{tabular}{|l|l|}
\hline
\textbf{Exemplo de Entrada}                                   & \textbf{Exemplo de Saída}                                              \\ \hline
\begin{tabular}[c]{@{}l@{}}1\\ 2\\ 3\\ -1\end{tabular}        & \begin{tabular}[c]{@{}l@{}}1\\ 2\\ -1\\ 2.00\\ 2\end{tabular}          \\ \hline
\begin{tabular}[c]{@{}l@{}}1\\ 1\\ 4\\ 0\end{tabular}         & \begin{tabular}[c]{@{}l@{}}1\\ 1\\ 12\\ 2.00\\ 3\end{tabular}          \\ \hline
\begin{tabular}[c]{@{}l@{}}4\\ 5\\ 0\end{tabular}             & \begin{tabular}[c]{@{}l@{}}12\\ 1\\ 4.50\\ 5\end{tabular}              \\ \hline
\begin{tabular}[c]{@{}l@{}}6\\ 7\\ -6\end{tabular}            & \begin{tabular}[c]{@{}l@{}}14\\ 1\\ 6.50\\ 13\end{tabular}             \\ \hline
\begin{tabular}[c]{@{}l@{}}10\\ 9\\ 8\\ 7\\ -185\end{tabular} & \begin{tabular}[c]{@{}l@{}}550\\ 25\\ 168\\ 1\\ 8.50\\ 55\end{tabular} \\ \hline
\end{tabular}
\caption{Questão E}
\end{table}

\newpage % Questão A 
\begin{center}
\textbf{{\Large Questão F - Duplas de Inteiros}}
\end{center}
\vspace{5pt}
Faça um programa que leia uma sequência de duplas de números inteiros do teclado: \textbf{\textit{A}} e \textbf{\textit{N}}. A quantidade de duplas da sequência é desconhecida, mas ela termina quando \textbf{\textit{A}} for igual a -1. A dupla que contém \textbf{A = -1} não faz parte da sequência, devendo ser desconsiderada.
\newline \newline
\textbf{{\large Entrada}} \newline
A entrada consiste de várias duplas de inteiros \textbf{\textit{A}} e \textbf{\textit{N}}, separados por espaço. Considere que pelo menos uma dupla válida será lida.
\newline \newline
\textbf{{\large Saída}} \newline
Ao final da leitura o programa deve imprimir, nessa ordem, a soma de todos os \textbf{N} que fazem dupla com \textbf{A} múltiplos de 8; a média de todos os \textbf{N} maiores que 3 (com duas casas decimais após a vírgula) e a soma da maior dupla da sequência, conforme exemplo abaixo.
\newline
\begin{table}[H]
\centering
\begin{tabular}{|l|l|}
\hline
\textbf{Exemplo de Entrada}                                       & \textbf{Exemplo de Saída}                             \\ \hline
\begin{tabular}[c]{@{}l@{}}1 -1\\ 8 2\\ 5 3\\ -1 4\end{tabular}   & \begin{tabular}[c]{@{}l@{}}2\\ 0.00\\ 10\end{tabular} \\ \hline
\begin{tabular}[c]{@{}l@{}}0 4\\ 8 5\\ -8 -4\\ -1 0\end{tabular}  & \begin{tabular}[c]{@{}l@{}}5\\ 4.50\\ 13\end{tabular} \\ \hline
\begin{tabular}[c]{@{}l@{}}8 -1\\ 16 1\\ 0 0\\ -1 25\end{tabular} & \begin{tabular}[c]{@{}l@{}}0\\ 0.00\\ 17\end{tabular} \\ \hline
\begin{tabular}[c]{@{}l@{}}0 4\\ -1 5\end{tabular}                & \begin{tabular}[c]{@{}l@{}}4\\ 4.00\\ 4\end{tabular}  \\ \hline
\begin{tabular}[c]{@{}l@{}}8 2\\ 4 3\\ -1 12\end{tabular}         & \begin{tabular}[c]{@{}l@{}}2\\ 0.00\\ 10\end{tabular} \\ \hline
\end{tabular}
\caption{Questão F}
\end{table}

\end{document}