\documentclass[a4paper, 12pt]{article}

\usepackage[top=2cm, bottom=2cm, left=2.5cm, right=2.5cm]{geometry}
\usepackage[utf8]{inputenc}
\usepackage{amsmath, amsfonts, amssymb}
\usepackage{graphicx} % inserir figuras - \includegraphics[scale=•]{•}
\usepackage{float} % ignorar regras de tipografia e inserir figura aonde queremos.
\usepackage[brazil]{babel} % Trocar Figure para Figura.
\usepackage{indentfirst}
\pagestyle{empty}


\begin{document}
\begin{figure}[H]
	\includegraphics[scale=0.9]{UnB_CiC_Logo.jpg}
\end{figure}
\noindent\rule{\textwidth}{0.4pt}
\begin{center}
	\textbf{{\Large Introdução à Ciência da Computação - 113913}} \newline \newline
	\textbf{{\large Prova 1} \\
	\vspace{9pt}
	{\large Questão A}} \\
	\noindent\rule{\textwidth}{0.4pt}
	\newline
\end{center}

\textbf{{\large Observações:}}
\begin{itemize}
	\item As provas também serão corrigidas por um \textbf{corretor automático}, portanto é necessário que as entradas e saídas do seu programa estejam conforme o padrão especificado em cada questão (exemplo de entrada e saída). Por exemplo, não use mensagens escritas durante o desenvolvimento do seu código como “Informe a primeira entrada”. Estas mensagens não são tratadas pelo corretor, portanto a correção irá resultar em resposta errada, mesmo que seu código esteja correto.
	\item Serão testadas várias entradas além das que foram dadas como exemplo, assim como as listas.
	\item Assim como as listas, as provas devem ser feitas na versão Python 3 ou superior.
	\item \textbf{Questão A valerá 30\% da nota da Prova 1 e a Questão B valerá 70\% da nota da Prova 1}.
	\item Leia com atenção e faça \textbf{exatamente} o que está sendo pedido.
\end{itemize}
\newpage % Questão A 
\begin{center}
\textbf{{\Large Questão A - Precipitação e vazão em bacias hidrográficas}}
\end{center}
\vspace{5pt}

Para fins de estudos hidrológicos, uma Bacia Hidrográfica (BH) é considerada um conjunto de terras drenadas por um corpo d’agua principal e seus afluentes. Vários autores consideram a importância do uso de conceito de BH similar ao de ecossistema, tanto para estudos como para o Gerenciamento Ambiental. A precipitação, a cobertura florestal e o tipo de solo/subsolo influenciam na vazão dos rios de uma BH. Analise os dados e diga a BH com maior média de índice pluviométrico anual, a BH com o período de menor índice pluviométrico e a área média de uma BH. 
\newline \newline
\textbf{{\large Entrada}} \newline

Várias linhas com nome da BH, área em Km2 (float), índice pluviométrico no verão, outono, inverno e primavera (inteiro).
A entrada termina quando o nome, a área e todos os índices da BH forem 0.
\newline \newline
\textbf{{\large Saída}} \newline
Nome da BH com maior média de índice pluviométrico anual.
\newline
Nome da BH com o período de menor índice pluviométrico.
\newline
A área média de uma BH em Km2 com 2 casas decimais de precisão.

\begin{table}[H]
\centering
\begin{tabular}{|1|1|}
\hline
\textbf{Exemplo de Entrada} & \textbf{Exemplo de Saída} \\ \hline
Ariquemes-RO 89766.12 120 400 800 70  & Cabixi-RO
\\
Cabixi-RO 15342.00 20 399 203 900     & Ji-Parana-RO
\\
Ji-Parana-RO 123928.10 293 12 933 100 & 89607.05
\\
Cajubim-RO 129392.00 333 112 300 78   &
\\
0 0 0 0 0 0 &
\\ \hline
\end{tabular}
\caption{Questão A}
\end{table}
\flushright
\textbf{\Large Boa Prova!}
\end{document}