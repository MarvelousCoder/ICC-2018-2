\documentclass[a4paper, 12pt]{article}

\usepackage[top=2cm, bottom=2cm, left=2.5cm, right=2.5cm]{geometry}
\usepackage[utf8]{inputenc}
\usepackage{amsmath, amsfonts, amssymb}
\usepackage{graphicx} % inserir figuras - \includegraphics[scale=•]{•}
\usepackage{float} % ignorar regras de tipografia e inserir figura aonde queremos.
\usepackage[brazil]{babel} % Trocar Figure para Figura.
\usepackage{indentfirst}
\pagestyle{empty}


\begin{document}
\begin{figure}[H]
	\includegraphics[scale=0.9]{UnB_CiC_Logo.jpg}
\end{figure}
\noindent\rule{\textwidth}{0.4pt}
\begin{center}
	\textbf{{\Large Introdução à Ciência da Computação - 113913}} \newline \newline
	\textbf{{\large Prova 1} \\
	\vspace{9pt}
	{\large Questão B}} \\
	\noindent\rule{\textwidth}{0.4pt}
	\newline
\end{center}

\textbf{{\large Observações:}}
\begin{itemize}
	\item As provas também serão corrigidas por um \textbf{corretor automático}, portanto é necessário que as entradas e saídas do seu programa estejam conforme o padrão especificado em cada questão (exemplo de entrada e saída). Por exemplo, não use mensagens escritas durante o desenvolvimento do seu código como “Informe a primeira entrada”. Estas mensagens não são tratadas pelo corretor, portanto a correção irá resultar em resposta errada, mesmo que seu código esteja correto.
	\item Serão testadas várias entradas além das que foram dadas como exemplo, assim como as listas.
	\item Assim como as listas, as provas devem ser feitas na versão Python 3 ou superior.
	\item \textbf{Questão A valerá 30\% da nota da Prova 1 e a Questão B valerá 70\% da nota da Prova 1}.
	\item Leia com atenção e faça \textbf{exatamente} o que está sendo pedido.
\end{itemize}
\newpage % Questão A 
\begin{center}
\textbf{{\Large Questão B - Fibonacci}}
\end{center}
\vspace{5pt}
Leia uma sequência de inteiros positivos do teclado, um por linha. A sequência termina quando for lido um inteiro menor ou igual a 0 (que não fará parte da sequência de números lidos). Para cada número 
\textbf{k} $>$ \textbf{0} lido, calcule o \textbf{k-ésimo}\,($F_k$) elemento da sequência de Fibonacci, conforme definição dada abaixo:
$$F_n =
		\begin{cases}
			1;\, n = 1\ \textrm{ou}\ n = 2 \\
			F_{n-1} + F_{n-2};\, n > 2 \\
		\end{cases}
$$
\newline \newline
\textbf{{\large Entrada}} \newline
Cada linha de entrada conterá um inteiro \textbf{\textit{k}}, quando a linha conter 
$k\, \leq \, 0$ o programa deve parar. Considere que pelo menos um \textbf{k} $>$ \textbf{0} será lido.
\newline \newline
\textbf{{\large Saída}} \newline
Considerando o valor de $F_k$:
\begin{itemize}
\item Caso $F_k$ seja par e k seja par, imprima a soma dos dois.
\item Caso $F_k$ seja par e k seja ímpar, imprima a diferença de $F_k$ com k.
\item Caso $F_k$ seja ímpar e k par, imprima a multiplicação.
\item Caso $F_k$ seja ímpar e k seja ímpar, imprima a divisão inteira de $F_k$ por k.
\end{itemize}
Ao final, informa a média aritmética dos números lidos da sequência com duas casas decimais e o maior $F_k$ calculado, conforme exemplo abaixo.
\newline
\begin{table}[H]
\centering
\begin{tabular}{|l|l|}
\hline
\textbf{Exemplo de Entrada}                                   & \textbf{Exemplo de Saída}                                              \\ \hline
\begin{tabular}[c]{@{}l@{}}1\\ 2\\ 3\\ -1\end{tabular}        & \begin{tabular}[c]{@{}l@{}}1\\ 2\\ -1\\ 2.00\\ 2\end{tabular}          \\ \hline
\begin{tabular}[c]{@{}l@{}}1\\ 1\\ 4\\ 0\end{tabular}         & \begin{tabular}[c]{@{}l@{}}1\\ 1\\ 12\\ 2.00\\ 3\end{tabular}          \\ \hline
\begin{tabular}[c]{@{}l@{}}4\\ 5\\ 0\end{tabular}             & \begin{tabular}[c]{@{}l@{}}12\\ 1\\ 4.50\\ 5\end{tabular}              \\ \hline
\begin{tabular}[c]{@{}l@{}}6\\ 7\\ -6\end{tabular}            & \begin{tabular}[c]{@{}l@{}}14\\ 1\\ 6.50\\ 13\end{tabular}             \\ \hline
\begin{tabular}[c]{@{}l@{}}10\\ 9\\ 8\\ 7\\ -185\end{tabular} & \begin{tabular}[c]{@{}l@{}}550\\ 25\\ 168\\ 1\\ 8.50\\ 55\end{tabular} \\ \hline
\end{tabular}
\caption{Questão B}
\end{table}
\flushright
\textbf{\Large Boa Prova!}
\end{document}