\documentclass[a4paper, 12pt]{article}

\usepackage[top=2cm, bottom=2cm, left=2.5cm, right=2.5cm]{geometry}
\usepackage[utf8]{inputenc}
\usepackage{amsmath, amsfonts, amssymb}
\usepackage{graphicx} % inserir figuras - \includegraphics[scale=•]{•}
\usepackage{float} % ignorar regras de tipografia e inserir figura aonde queremos.
\usepackage[brazil]{babel} % Trocar Figure para Figura.
\usepackage{indentfirst}
\pagestyle{empty}


\begin{document}
\begin{figure}[H]
	\includegraphics[scale=0.9]{UnB_CiC_Logo.jpg}
\end{figure}
\noindent\rule{\textwidth}{0.4pt}
\begin{center}
	\textbf{{\Large Introdução à Ciência da Computação - 113913}} \newline \newline
	\textbf{{\large Prova 1} \\
	\vspace{9pt}
	{\large Questão B}} \\
	\noindent\rule{\textwidth}{0.4pt}
	\newline
\end{center}

\textbf{{\large Observações:}}
\begin{itemize}
	\item As provas também serão corrigidas por um \textbf{corretor automático}, portanto é necessário que as entradas e saídas do seu programa estejam conforme o padrão especificado em cada questão (exemplo de entrada e saída). Por exemplo, não use mensagens escritas durante o desenvolvimento do seu código como “Informe a primeira entrada”. Estas mensagens não são tratadas pelo corretor, portanto a correção irá resultar em resposta errada, mesmo que seu código esteja correto.
	\item Serão testadas várias entradas além das que foram dadas como exemplo, assim como as listas.
	\item Assim como as listas, as provas devem ser feitas na versão Python 3 ou superior.
	\item \textbf{Questão A valerá 30\% da nota da Prova 1 e a Questão B valerá 70\% da nota da Prova 1}.
	\item Leia com atenção e faça \textbf{exatamente} o que está sendo pedido.
\end{itemize}
\newpage % Questão A 
\begin{center}
\textbf{{\Large Questão B - Sequência Fibonacci}}
\end{center}
\vspace{5pt}
Faça um programa que leia um valor \textbf{N}, depois leia \textbf{N} vezes valores \textbf{k}. Dada a definição abaixo:
$$F_n =
		\begin{cases}
			1;\, n = 1\ \textrm{ou}\ n = 2 \\
			F_{n-1} + F_{n-2};\, n > 2 \\
		\end{cases}
$$ 
Faça o cálculo do k-ésimo elemento ($F_k$) da sequência de Fibonacci.
\newline \newline
\textbf{{\large Entrada}} \newline
A entrada consiste apenas de inteiros \textbf{N} e \textbf{k}. A primeira linha conterá um \textbf{N} $\geq$ \textbf{0}. Nas próximas \textbf{N} linhas serão lidos valores \textbf{k}.
\newline \newline
\textbf{{\large Saída}} \newline
Caso \textbf{k} seja maior que 0, imprima na tela o \textbf{k-ésimo} elemento da sequência de Fibonacci. Caso esse \textbf{k-ésimo} elemento seja ímpar, imprima também $F_{k+1}$. Caso \textbf{k} seja menor ou igual a zero, apenas imprima na tela a mensagem ``erro''. Ao final, informe a média (com 2 casas decimais após a vírgula) de todos os \textbf{k} lidos.
\newline
\begin{table}[H]
\centering
\begin{tabular}{|l|l|}
\hline
\textbf{Exemplo de Entrada}                           & \textbf{Exemplo de Saída}                                           \\ \hline
\begin{tabular}[c]{@{}l@{}}3\\ 1\\ 2\\ 3\end{tabular} & \begin{tabular}[c]{@{}l@{}}1\\ 1\\ 1\\ 2\\ 2\\ 2.00\end{tabular}    \\ \hline
\begin{tabular}[c]{@{}l@{}}3\\ 1\\ 4\\ 0\end{tabular} & \begin{tabular}[c]{@{}l@{}}1\\ 1\\ 3\\ 5\\ erro\\ 1.67\end{tabular} \\ \hline
\begin{tabular}[c]{@{}l@{}}2\\ -1\\ -2\end{tabular}   & \begin{tabular}[c]{@{}l@{}}erro\\ erro\\ -1.50\end{tabular}         \\ \hline
\begin{tabular}[c]{@{}l@{}}1\\ 6\end{tabular}         & \begin{tabular}[c]{@{}l@{}}8\\ 6.00\end{tabular}                    \\ \hline
\end{tabular}
\caption{Questão B}
\end{table}
\flushright
\textbf{\Large Boa Prova!}
\end{document}