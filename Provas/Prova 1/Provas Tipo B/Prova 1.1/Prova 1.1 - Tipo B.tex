\documentclass[a4paper, 12pt]{article}

\usepackage[top=2cm, bottom=2cm, left=2.5cm, right=2.5cm]{geometry}
\usepackage[utf8]{inputenc}
\usepackage{amsmath, amsfonts, amssymb}
\usepackage{graphicx} % inserir figuras - \includegraphics[scale=•]{•}
\usepackage{float} % ignorar regras de tipografia e inserir figura aonde queremos.
\usepackage[brazil]{babel} % Trocar Figure para Figura.
\usepackage{indentfirst}
\pagestyle{empty}


\begin{document}
\begin{figure}[H]
	\includegraphics[scale=0.9]{UnB_CiC_Logo.jpg}
\end{figure}
\noindent\rule{\textwidth}{0.4pt}
\begin{center}
	\textbf{{\Large Introdução à Ciência da Computação - 113913}} \newline \newline
	\textbf{{\large Prova 1} \\
	\vspace{9pt}
	{\large Questão B}} \\
	\noindent\rule{\textwidth}{0.4pt}
	\newline
\end{center}

\textbf{{\large Observações:}}
\begin{itemize}
	\item As provas também serão corrigidas por um \textbf{corretor automático}, portanto é necessário que as entradas e saídas do seu programa estejam conforme o padrão especificado em cada questão (exemplo de entrada e saída). Por exemplo, não use mensagens escritas durante o desenvolvimento do seu código como “Informe a primeira entrada”. Estas mensagens não são tratadas pelo corretor, portanto a correção irá resultar em resposta errada, mesmo que seu código esteja correto.
	\item Serão testadas várias entradas além das que foram dadas como exemplo, assim como as listas.
	\item Assim como as listas, as provas devem ser feitas na versão Python 3 ou superior.
	\item \textbf{Questão A valerá 30\% da nota da Prova 1 e a Questão B valerá 70\% da nota da Prova 1}.
	\item Leia com atenção e faça \textbf{exatamente} o que está sendo pedido.
\end{itemize}
\newpage % Questão A 
\begin{center}
\textbf{{\Large Questão B - Função Sigma e Tal}}
\end{center}
\vspace{5pt}
A função sigma denotada por $\sigma(n)$ é a função que soma os divisores distintos de \textbf{\textit{n}}, \textbf{incluindo 1 e n}. A função tal denotada por $\tau(n)$ é a função que retorna
a quantidade de divisores distintos de \textbf{\textit{n}}, \textbf{incluindo 1 e n}. 
\newline \newline
\textbf{{\large Entrada}} \newline
A entrada consiste de um inteiro \textbf{n}, onde $\textrm{\textbf{n}} \geq \textrm{\textbf{1}}$.
\newline \newline
\textbf{{\large Saída}} \newline
A saída será composta de 3 linhas: a primeira linha conterá todos os divisores de \textbf{\textit{n}} separados por espaço, em uma única linha, conforme exemplo abaixo. \textbf{Não deve haver espaços em branco após o último valor da linha}. A segunda linha será o valor $\sigma(n)$, e a terceira $\tau(n)$.
\newline \newline
\textbf{{\large Nota}} \newline
No primeiro exemplo, o número 4 tem três divisores: 1, 2 e 4. $\sigma(4) = 1 + 2 + 4 = 7$ e $\tau(4) = 3$.
\newline
\begin{table}[H]
\centering
\begin{tabular}{|l|l|}
\hline
\textbf{Exemplo de Entrada} & \textbf{Exemplo de Saída}                                                 \\ \hline
4                           & \begin{tabular}[c]{@{}l@{}}1 2 4\\ 7\\ 3\end{tabular}                     \\ \hline
5                           & \begin{tabular}[c]{@{}l@{}}1 5\\ 6\\ 2\end{tabular}                       \\ \hline
12                          & \begin{tabular}[c]{@{}l@{}}1 2 3 4 6 12\\ 28\\ 6\end{tabular}             \\ \hline
100                         & \begin{tabular}[c]{@{}l@{}}1 2 4 5 10 20 25 50 100\\ 217\\ 9\end{tabular} \\ \hline
50                          & \begin{tabular}[c]{@{}l@{}}1 2 5 10 25 50\\ 93\\ 6\end{tabular}           \\ \hline
\end{tabular}
\caption{Questão B}
\end{table}
\flushright
\textbf{\Large Boa Prova!}
\end{document}