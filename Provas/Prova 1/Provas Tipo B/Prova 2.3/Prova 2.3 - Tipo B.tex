\documentclass[a4paper, 12pt]{article}

\usepackage[top=2cm, bottom=2cm, left=2.5cm, right=2.5cm]{geometry}
\usepackage[utf8]{inputenc}
\usepackage{amsmath, amsfonts, amssymb}
\usepackage{graphicx} % inserir figuras - \includegraphics[scale=•]{•}
\usepackage{float} % ignorar regras de tipografia e inserir figura aonde queremos.
\usepackage[brazil]{babel} % Trocar Figure para Figura.
\usepackage{indentfirst}
\pagestyle{empty}


\begin{document}
\begin{figure}[H]
	\includegraphics[scale=0.9]{UnB_CiC_Logo.jpg}
\end{figure}
\noindent\rule{\textwidth}{0.4pt}
\begin{center}
	\textbf{{\Large Introdução à Ciência da Computação - 113913}} \newline \newline
	\textbf{{\large Prova 1} \\
	\vspace{9pt}
	{\large Questão B}} \\
	\noindent\rule{\textwidth}{0.4pt}
	\newline
\end{center}

\textbf{{\large Observações:}}
\begin{itemize}
	\item As provas também serão corrigidas por um \textbf{corretor automático}, portanto é necessário que as entradas e saídas do seu programa estejam conforme o padrão especificado em cada questão (exemplo de entrada e saída). Por exemplo, não use mensagens escritas durante o desenvolvimento do seu código como “Informe a primeira entrada”. Estas mensagens não são tratadas pelo corretor, portanto a correção irá resultar em resposta errada, mesmo que seu código esteja correto.
	\item Serão testadas várias entradas além das que foram dadas como exemplo, assim como as listas.
	\item Assim como as listas, as provas devem ser feitas na versão Python 3 ou superior.
	\item \textbf{Questão A valerá 30\% da nota da Prova 1 e a Questão B valerá 70\% da nota da Prova 1}.
	\item Leia com atenção e faça \textbf{exatamente} o que está sendo pedido.
\end{itemize}
\newpage % Questão A 
\begin{center}
\textbf{{\Large Questão B - Sequência Par}}
\end{center}
\vspace{5pt}
Leia uma sequência de duplas de inteiros \textbf{X} e \textbf{Y} do teclado. A quantidade de duplas da sequência é desconhecida, mas ela termina quando \textbf{Y} for menor que 0. A dupla que contém \textbf{Y} $<$ \textbf{0} não faz parte da sequência, devendo ser desconsiderada.
\newline \newline
\textbf{{\large Entrada}} \newline
A entrada consiste apenas de inteiros, onde cada linha de entrada contém dois inteiros \textbf{X} e \textbf{Y}, separados por espaço. A linha que conter \textbf{Y} $<$ \textbf{0} deverá ser desconsiderada. Considere que pelo menos uma dupla válida será lida.
\newline \newline
\textbf{{\large Saída}} \newline
O programa deve imprimir na tela a soma \textbf{S} de \textbf{Y} pares consecutivos a partir de \textbf{X} inclusive o próprio \textbf{X}, se ele for par. Por exemplo, para a entrada 4 5, a saída deve ser 40, que é equivalente à: 4 + 6 + 8 + 10 + 12. No final, imprima também a maior e a menor soma \textbf{S}, e a média das somas \textbf{S} (com duas casas decimais após a vírgula).
\newline
\begin{table}[H]
\centering
\begin{tabular}{|l|l|}
\hline
\textbf{Exemplo de Entrada}                                          & \textbf{Exemplo de Saída}                                                 \\ \hline
\begin{tabular}[c]{@{}l@{}}4 5\\ 3 2\\ 1 0\\ 1 -1\end{tabular}       & \begin{tabular}[c]{@{}l@{}}40\\ 10\\ 0\\ 40\\ 0\\ 16.67\end{tabular}      \\ \hline
\begin{tabular}[c]{@{}l@{}}-5 1\\ -3 2\\ -10 3\\ -10 -3\end{tabular} & \begin{tabular}[c]{@{}l@{}}-4\\ -2\\ -24\\ -2\\ -24\\ -10.00\end{tabular} \\ \hline
\begin{tabular}[c]{@{}l@{}}3 3\\ 2 2\\ -1 2\\ 2 -2\end{tabular}      & \begin{tabular}[c]{@{}l@{}}18\\ 6\\ 2\\ 18\\ 2\\ 8.67\end{tabular}        \\ \hline
\begin{tabular}[c]{@{}l@{}}1 4\\ -1 -4\end{tabular}                  & \begin{tabular}[c]{@{}l@{}}20\\ 20\\ 20\\ 20.00\end{tabular}              \\ \hline
\end{tabular}
\caption{Questão B}
\end{table}
\flushright
\textbf{\Large Boa Prova!}
\end{document}