\documentclass[a4paper, 12pt]{article}

\usepackage[top=2cm, bottom=2cm, left=2.5cm, right=2.5cm]{geometry}
\usepackage[utf8]{inputenc}
\usepackage{amsmath, amsfonts, amssymb}
\usepackage{graphicx} % inserir figuras - \includegraphics[scale=•]{•}
\usepackage{float} % ignorar regras de tipografia e inserir figura aonde queremos.
\usepackage[brazil]{babel} % Trocar Figure para Figura.
\usepackage{indentfirst}
\pagestyle{empty}


\begin{document}
\begin{figure}[H]
	\includegraphics[scale=0.9]{UnB_CiC_Logo.jpg}
\end{figure}
\noindent\rule{\textwidth}{0.4pt}
\begin{center}
	\textbf{{\Large Introdução à Ciência da Computação - 113913}} \newline \newline
	\textbf{{\large Prova 1} \\
	\vspace{9pt}
	{\large Questão B}} \\
	\noindent\rule{\textwidth}{0.4pt}
	\newline
\end{center}

\textbf{{\large Observações:}}
\begin{itemize}
	\item As provas também serão corrigidas por um \textbf{corretor automático}, portanto é necessário que as entradas e saídas do seu programa estejam conforme o padrão especificado em cada questão (exemplo de entrada e saída). Por exemplo, não use mensagens escritas durante o desenvolvimento do seu código como “Informe a primeira entrada”. Estas mensagens não são tratadas pelo corretor, portanto a correção irá resultar em resposta errada, mesmo que seu código esteja correto.
	\item Serão testadas várias entradas além das que foram dadas como exemplo, assim como as listas.
	\item Assim como as listas, as provas devem ser feitas na versão Python 3 ou superior.
	\item \textbf{Questão A valerá 30\% da nota da Prova 1 e a Questão B valerá 70\% da nota da Prova 1}.
	\item Leia com atenção e faça \textbf{exatamente} o que está sendo pedido.
\end{itemize}
\newpage % Questão A 
\begin{center}
\textbf{{\Large Questão B - Mínimo Múltiplo Comum}}
\end{center}
\vspace{5pt}
O mínimo múltiplo comum (mmc) de dois inteiros a e b é o menor inteiro positivo que é múltiplo simultaneamente de a e de b. Se não existir tal inteiro positivo, por exemplo, se a = 0 ou b = 0, então \textit{\textbf{mmc(a,b)}} é zero por definição. Sabemos que $a\cdot b = mmc(a,b) \cdot mdc(a,b)$.
\newline \newline
\textbf{{\large Entrada}} \newline
A entrada contém apenas valores inteiros, sendo \textbf{\textit{N}} $>$ \textbf{\textit{0}} e \textbf{\textit{A,B}} $\geq$ \textbf{\textit{0}}. Na primeira linha será lido o valor \textbf{N} e nas próximas \textbf{N} linhas serão lidos os valores \textbf{A} e \textbf{B}, separados por espaço.
\newline \newline
\textbf{{\large Saída}} \newline
Para cada valor \textbf{\textit{A}} e \textbf{\textit{B}} lidos, calcule e imprima seu mmc. Ao final, imprima a média dos mínimos múltiplos comuns (com duas casas decimais após a vírgula) dos mmcs calculados.
\newline
\begin{table}[H]
\centering
\begin{tabular}{|l|l|}
\hline
\textbf{Exemplo de Entrada}                                       & \textbf{Exemplo de Saída}                             \\ \hline
\begin{tabular}[c]{@{}l@{}}2 \\ 4 8\\ 3 5\end{tabular}   & \begin{tabular}[c]{@{}l@{}}8\\ 15\\ 11.50 \end{tabular} \\ \hline
\begin{tabular}[c]{@{}l@{}}2 \\ 0 5\\ 5 0\end{tabular}  & \begin{tabular}[c]{@{}l@{}}0\\ 0\\ 0.00\end{tabular} \\ \hline
\begin{tabular}[c]{@{}l@{}}3 \\ 12 8 \\ 3 4\\ 0 2\end{tabular} & \begin{tabular}[c]{@{}l@{}}24\\ 12 \\ 0 \\ 12.00\end{tabular} \\ \hline
\begin{tabular}[c]{@{}l@{}}1 \\ 8 24\end{tabular} & \begin{tabular}[c]{@{}l@{}}24\\ 24.00\end{tabular} \\ \hline
\begin{tabular}[c]{@{}l@{}}2 \\ 4 2\\8 10\end{tabular} & \begin{tabular}[c]{@{}l@{}}4\\ 40 \\ 22.00\end{tabular} \\ \hline
\end{tabular}
\caption{Questão B}
\end{table}
\flushright
\textbf{\Large Boa Prova!}
\end{document}