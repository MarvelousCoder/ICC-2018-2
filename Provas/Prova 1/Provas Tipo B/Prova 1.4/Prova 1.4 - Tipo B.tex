\documentclass[a4paper, 12pt]{article}

\usepackage[top=2cm, bottom=2cm, left=2.5cm, right=2.5cm]{geometry}
\usepackage[utf8]{inputenc}
\usepackage{amsmath, amsfonts, amssymb}
\usepackage{graphicx} % inserir figuras - \includegraphics[scale=•]{•}
\usepackage{float} % ignorar regras de tipografia e inserir figura aonde queremos.
\usepackage[brazil]{babel} % Trocar Figure para Figura.
\usepackage{indentfirst}
\pagestyle{empty}


\begin{document}
\begin{figure}[H]
	\includegraphics[scale=0.9]{UnB_CiC_Logo.jpg}
\end{figure}
\noindent\rule{\textwidth}{0.4pt}
\begin{center}
	\textbf{{\Large Introdução à Ciência da Computação - 113913}} \newline \newline
	\textbf{{\large Prova 1} \\
	\vspace{9pt}
	{\large Questão B}} \\
	\noindent\rule{\textwidth}{0.4pt}
	\newline
\end{center}

\textbf{{\large Observações:}}
\begin{itemize}
	\item As provas também serão corrigidas por um \textbf{corretor automático}, portanto é necessário que as entradas e saídas do seu programa estejam conforme o padrão especificado em cada questão (exemplo de entrada e saída). Por exemplo, não use mensagens escritas durante o desenvolvimento do seu código como “Informe a primeira entrada”. Estas mensagens não são tratadas pelo corretor, portanto a correção irá resultar em resposta errada, mesmo que seu código esteja correto.
	\item Serão testadas várias entradas além das que foram dadas como exemplo, assim como as listas.
	\item Assim como as listas, as provas devem ser feitas na versão Python 3 ou superior.
	\item \textbf{Questão A valerá 30\% da nota da Prova 1 e a Questão B valerá 70\% da nota da Prova 1}.
	\item Leia com atenção e faça \textbf{exatamente} o que está sendo pedido.
\end{itemize}
\newpage % Questão A 
\begin{center}
\textbf{{\Large Questão B - The Winter is Coming}}
\end{center}
\vspace{5pt}
Os Starks sempre avisaram: ``The Winter is Coming'' e o inverno finalmente chegou em Westeros. O Rei do Norte, Jon Snow, decidiu igualar o ouro entre todas as casas do Norte, dando ouro para algumas. Para isso, ele pediu para você, o Mestre da Moeda, considerar o ouro (em kg) que cada uma possui e calcular o custo mínimo do presente do rei, sabendo que: no Norte existem \textbf{\textit{n}} casas, o ouro que cada uma possui é estimado em um inteiro $a_i$ e que o rei apenas dará ouro, não tirará de ninguém.
\newline \newline
\textbf{{\large Entrada}} \newline
A primeira linha contém um inteiro \textbf{\textit{n}}
 $(\textrm{\textbf{1}}\, \leq \textrm{\textbf{n}}\, \leq \textrm{\textbf{100}})$ - o número de casas no Norte.
As próximas \textbf{\textit{n}} linhas contém os inteiros $a_1$, $a_2$, $a_3$, \dots, $a_n$, onde 
$a_i \geq 0$ corresponde ao ouro, em kg, que cada casa possui. Considere que o primeiro inteiro $a_i$ sempre será o ouro correspondente da casa que \textbf{possui mais ouro}.
\newline \newline
\textbf{{\large Saída}} \newline
Um único inteiro que corresponde a quantidade mínima de ouro (em kg) que Winterfell irá gastar para que todas as casas tenham a mesma quantidade de ouro.
\newline \newline
\textbf{{\large Nota}} \newline
No primeiro exemplo se adicionarmos para a segunda casa 4 kg, para a terceira 3 e para a quarta 2, então todas elas terão 4 kg. \newline
No quarto exemplo não é possível dar nada para ninguém, porque todas as casas possuem 12 kg.
\begin{table}[H]
\centering
\begin{tabular}{|l|l|}
\hline
\textbf{Exemplo de Entrada}                               & \textbf{Exemplo de Saída} \\ \hline
\begin{tabular}[c]{@{}l@{}}4\\ 4\\ 0\\ 1\\ 2\end{tabular} & 9                         \\ \hline
\begin{tabular}[c]{@{}l@{}}3\\ 1\\ 1\\ 0\end{tabular}     & 1                         \\ \hline
\begin{tabular}[c]{@{}l@{}}2\\ 3\\ 1\end{tabular}         & 2                         \\ \hline
\begin{tabular}[c]{@{}l@{}}1\\ 12\end{tabular}            & 0                         \\ \hline
\end{tabular}
\caption{Questão B}
\end{table}
\flushright
\textbf{\Large Boa Prova!}
\end{document}