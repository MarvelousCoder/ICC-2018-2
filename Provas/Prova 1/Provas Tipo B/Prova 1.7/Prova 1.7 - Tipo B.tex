\documentclass[a4paper, 12pt]{article}

\usepackage[top=2cm, bottom=2cm, left=2.5cm, right=2.5cm]{geometry}
\usepackage[utf8]{inputenc}
\usepackage{amsmath, amsfonts, amssymb}
\usepackage{graphicx} % inserir figuras - \includegraphics[scale=•]{•}
\usepackage{float} % ignorar regras de tipografia e inserir figura aonde queremos.
\usepackage[brazil]{babel} % Trocar Figure para Figura.
\usepackage{indentfirst}
\pagestyle{empty}


\begin{document}
\begin{figure}[H]
	\includegraphics[scale=0.9]{UnB_CiC_Logo.jpg}
\end{figure}
\noindent\rule{\textwidth}{0.4pt}
\begin{center}
	\textbf{{\Large Introdução à Ciência da Computação - 113913}} \newline \newline
	\textbf{{\large Prova 1} \\
	\vspace{9pt}
	{\large Questão B}} \\
	\noindent\rule{\textwidth}{0.4pt}
	\newline
\end{center}

\textbf{{\large Observações:}}
\begin{itemize}
	\item As provas também serão corrigidas por um \textbf{corretor automático}, portanto é necessário que as entradas e saídas do seu programa estejam conforme o padrão especificado em cada questão (exemplo de entrada e saída). Por exemplo, não use mensagens escritas durante o desenvolvimento do seu código como “Informe a primeira entrada”. Estas mensagens não são tratadas pelo corretor, portanto a correção irá resultar em resposta errada, mesmo que seu código esteja correto.
	\item Serão testadas várias entradas além das que foram dadas como exemplo, assim como as listas.
	\item Assim como as listas, as provas devem ser feitas na versão Python 3 ou superior.
	\item \textbf{Questão A valerá 30\% da nota da Prova 1 e a Questão B valerá 70\% da nota da Prova 1}.
	\item Leia com atenção e faça \textbf{exatamente} o que está sendo pedido.
\end{itemize}
\newpage % Questão A 
\begin{center}
\textbf{{\Large Questão B - Máximo Divisor Comum}}
\end{center}
\vspace{5pt}
O máximo divisor comum entre dois ou mais números inteiros é o maior número inteiro que é fator de tais números. Por exemplo, os divisores comuns de 12 e 18 são 1, 2, 3 e 6, logo \textbf{\textit{mdc(12,18) = 6}}. Dizemos que dois números inteiros a e b são primos entre si, se e somente se \textbf{\textit{mdc(a,b) = 1}}. Faça um programa que leia uma sequência de duplas de inteiros do teclado, \textbf{A} e \textbf{B}. A quantidade de duplas da sequência é desconhecida, mas ela termina quando \textbf{A} ou \textbf{B} for menor ou igual a zero. A dupla que contém \textbf{A} ou \textbf{B} menor ou igual a zero não faz parte da sequência, devendo ser desconsiderada.
\newline \newline
\textbf{{\large Entrada}} \newline
A entrada será a sequência de duplas de inteiros, cada linha de entrada contém dois inteiros \textbf{A} e \textbf{B}, separados por espaço. Considere que a sequência contém pelo menos uma dupla.
\newline \newline
\textbf{{\large Saída}} \newline
Para cada \textbf{A} e \textbf{B} lidos que fazem parte da sequência, calcule e imprima na tela \textbf{\textit{mdc(A,B)}}. Ao final imprima a média de todos os máximos divisores comuns calculados com duas casas decimais após a vírgula.
\newline
\begin{table}[H]
\centering
\begin{tabular}{|l|l|}
\hline
\textbf{Exemplo de Entrada}                                       & \textbf{Exemplo de Saída}                             \\ \hline
\begin{tabular}[c]{@{}l@{}}8 12\\ 7 9\\ 397 311\\ -0 4\end{tabular}   & \begin{tabular}[c]{@{}l@{}}4\\ 1\\ 1 \\ 2.00\end{tabular} \\ \hline
\begin{tabular}[c]{@{}l@{}}8 13\\ 8 14\\ 4 0\end{tabular}  & \begin{tabular}[c]{@{}l@{}}1\\ 2\\ 1.50\end{tabular} \\ \hline
\begin{tabular}[c]{@{}l@{}}16 120 \\ -1 -1\end{tabular} & \begin{tabular}[c]{@{}l@{}}8\\ 8.00 \end{tabular} \\ \hline
\end{tabular}
\caption{Questão B}
\end{table}
\flushright
\textbf{\Large Boa Prova!}
\end{document}