\documentclass[a4paper, 12pt]{article}

\usepackage[top=2cm, bottom=2cm, left=2.5cm, right=2.5cm]{geometry}
\usepackage[utf8]{inputenc}
\usepackage{amsmath, amsfonts, amssymb}
\usepackage{graphicx} % inserir figuras - \includegraphics[scale=•]{•}
\usepackage{float} % ignorar regras de tipografia e inserir figura aonde queremos.
\usepackage[brazil]{babel} % Trocar Figure para Figura.
\usepackage{indentfirst}
\pagestyle{empty}


\begin{document}
\begin{figure}[H]
	\includegraphics[scale=0.9]{UnB_CiC_Logo.jpg}
\end{figure}
\noindent\rule{\textwidth}{0.4pt}
\begin{center}
	\textbf{{\Large Introdução à Ciência da Computação - 113913}} \newline \newline
	\textbf{{\large Prova 1} \\
	\vspace{9pt}
	{\large Questão B}} \\
	\noindent\rule{\textwidth}{0.4pt}
	\newline
\end{center}

\textbf{{\large Observações:}}
\begin{itemize}
	\item As provas também serão corrigidas por um \textbf{corretor automático}, portanto é necessário que as entradas e saídas do seu programa estejam conforme o padrão especificado em cada questão (exemplo de entrada e saída). Por exemplo, não use mensagens escritas durante o desenvolvimento do seu código como “Informe a primeira entrada”. Estas mensagens não são tratadas pelo corretor, portanto a correção irá resultar em resposta errada, mesmo que seu código esteja correto.
	\item Serão testadas várias entradas além das que foram dadas como exemplo, assim como as listas.
	\item Assim como as listas, as provas devem ser feitas na versão Python 3 ou superior.
	\item \textbf{Questão A valerá 30\% da nota da Prova 1 e a Questão B valerá 70\% da nota da Prova 1}.
	\item Leia com atenção e faça \textbf{exatamente} o que está sendo pedido.
\end{itemize}
\newpage % Questão A 
\begin{center}
\textbf{{\Large Questão B - Somatório Pares}}
\end{center}
\vspace{5pt}
Faça um programa que leia uma sequência de inteiros \textbf{n}, lidos do teclado. A quantidade de elementos da sequência é desconhecida, mas ela termina quando \textbf{n} for menor que 0, que não faz parte da sequência e deve ser desconsiderado.
\newline \newline
\textbf{{\large Entrada}} \newline
Cada linha de entrada conterá um inteiro \textbf{n}, a linha de entrada que conter \textbf{n} $<$ \textbf{0} deverá ser desconsiderada. Considere que a entrada terá pelo menos um \textbf{n} $\geq$ \textbf{0}.
\newline \newline
\textbf{{\large Saída}} \newline
Para cada \textbf{n} $\geq$ \textbf{0} lido imprima na tela a soma \textbf{S} de todos os números pares de 0 até \textbf{n}, incluindo o \textbf{n}, se for o caso. Ao final imprima a maior soma \textbf{S} e a média (com duas casas decimais após a vírgula) das somas \textbf{S} calculadas, conforme exemplo abaixo.
\newline
\begin{table}[H]
\centering
\begin{tabular}{|l|l|}
\hline
\textbf{Exemplo de Entrada}                            & \textbf{Exemplo de Saída}                                      \\ \hline
\begin{tabular}[c]{@{}l@{}}10\\ 4\\ -1\end{tabular}    & \begin{tabular}[c]{@{}l@{}}30\\ 6\\ 30\\ 18.00\end{tabular}    \\ \hline
\begin{tabular}[c]{@{}l@{}}6\\ 2\\ 4\\ -1\end{tabular} & \begin{tabular}[c]{@{}l@{}}12\\ 2\\ 6\\ 12\\ 6.67\end{tabular} \\ \hline
\begin{tabular}[c]{@{}l@{}}3\\ 4\\ 5\\ -8\end{tabular} & \begin{tabular}[c]{@{}l@{}}2\\ 6\\ 6\\ 6\\ 4.67\end{tabular}   \\ \hline
\begin{tabular}[c]{@{}l@{}}8\\ -1\end{tabular}         & \begin{tabular}[c]{@{}l@{}}20\\ 20\\ 20.00\end{tabular}        \\ \hline
\end{tabular}
\caption{Questão B}
\end{table}
\flushright
\textbf{\Large Boa Prova!}
\end{document}