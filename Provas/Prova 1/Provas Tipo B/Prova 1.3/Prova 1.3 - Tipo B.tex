\documentclass[a4paper, 12pt]{article}

\usepackage[top=2cm, bottom=2cm, left=2.5cm, right=2.5cm]{geometry}
\usepackage[utf8]{inputenc}
\usepackage{amsmath, amsfonts, amssymb}
\usepackage{graphicx} % inserir figuras - \includegraphics[scale=•]{•}
\usepackage{float} % ignorar regras de tipografia e inserir figura aonde queremos.
\usepackage[brazil]{babel} % Trocar Figure para Figura.
\usepackage{indentfirst}
\pagestyle{empty}


\begin{document}
\begin{figure}[H]
	\includegraphics[scale=0.9]{UnB_CiC_Logo.jpg}
\end{figure}
\noindent\rule{\textwidth}{0.4pt}
\begin{center}
	\textbf{{\Large Introdução à Ciência da Computação - 113913}} \newline \newline
	\textbf{{\large Prova 1} \\
	\vspace{9pt}
	{\large Questão B}} \\
	\noindent\rule{\textwidth}{0.4pt}
	\newline
\end{center}

\textbf{{\large Observações:}}
\begin{itemize}
	\item As provas também serão corrigidas por um \textbf{corretor automático}, portanto é necessário que as entradas e saídas do seu programa estejam conforme o padrão especificado em cada questão (exemplo de entrada e saída). Por exemplo, não use mensagens escritas durante o desenvolvimento do seu código como “Informe a primeira entrada”. Estas mensagens não são tratadas pelo corretor, portanto a correção irá resultar em resposta errada, mesmo que seu código esteja correto.
	\item Serão testadas várias entradas além das que foram dadas como exemplo, assim como as listas.
	\item Assim como as listas, as provas devem ser feitas na versão Python 3 ou superior.
	\item \textbf{Questão A valerá 30\% da nota da Prova 1 e a Questão B valerá 70\% da nota da Prova 1}.
	\item Leia com atenção e faça \textbf{exatamente} o que está sendo pedido.
\end{itemize}
\newpage % Questão A 
\begin{center}
\textbf{{\Large Questão B - Impressão de Sequência}}
\end{center}
\vspace{5pt}
Escreva um programa que leia dois valores inteiros \textbf{\textit{X}} e \textbf{\textit{Y}}. A seguir, mostre uma sequência de 1 até \textbf{\textit{Y}}, passando para a próxima linha a cada \textbf{\textit{X}} números e a soma de todos os números dessa sequência.
\newline \newline
\textbf{{\large Entrada}} \newline
A entrada contém duas linhas. A primeira linha será o \textbf{\textit{X}} e a segunda o \textbf{\textit{Y}}, onde \textbf{X e Y são maiores que 0}.
\newline \newline
\textbf{{\large Saída}} \newline
Cada sequência deve ser impressa em uma linha apenas, com 1 espaço em branco entre cada número, conforme exemplo abaixo. \textbf{Não deve haver espaço em branco após o último valor de cada linha}. No final, deverá ser impresso em uma nova linha a soma de todos os números mostrados na tela.
\newline
\begin{table}[H]
\centering
\begin{tabular}{|l|l|}
\hline
\textbf{Exemplo de Entrada}                    & \textbf{Exemplo de Saída}                                                                          \\ \hline
\begin{tabular}[c]{@{}l@{}}3\\ 99\end{tabular} & \begin{tabular}[c]{@{}l@{}}1 2 3\\ 4 5 6\\ 7 8 9\\ 10 11 12\\ . . .\\ 97 98 99\\ 4950\end{tabular} \\ \hline
\begin{tabular}[c]{@{}l@{}}2\\ 9\end{tabular} & \begin{tabular}[c]{@{}l@{}}1 2\\ 3 4\\ 5 6\\ 7 8\\ 9\\ 45\end{tabular}                             \\ \hline
\begin{tabular}[c]{@{}l@{}}4\\ 3\end{tabular}  & \begin{tabular}[c]{@{}l@{}}1 2 3\\ 6\end{tabular}                                                  \\ \hline
\end{tabular}
\caption{Questão B}
\end{table}
\flushright
\textbf{\Large Boa Prova!}
\end{document}