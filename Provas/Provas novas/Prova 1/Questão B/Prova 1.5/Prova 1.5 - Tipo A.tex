\documentclass[a4paper, 12pt]{article}

\usepackage[top=2cm, bottom=2cm, left=2.5cm, right=2.5cm]{geometry}
\usepackage[utf8]{inputenc}
\usepackage{amsmath, amsfonts, amssymb}
\usepackage{graphicx} % inserir figuras - \includegraphics[scale=•]{•}
\usepackage{float} % ignorar regras de tipografia e inserir figura aonde queremos.
\usepackage[brazil]{babel} % Trocar Figure para Figura.
\usepackage{indentfirst}
\pagestyle{empty}


\begin{document}
\begin{figure}[H]
	\includegraphics[scale=0.9]{UnB_CiC_Logo.jpg}
\end{figure}
\noindent\rule{\textwidth}{0.4pt}
\begin{center}
	\textbf{{\Large Introdução à Ciência da Computação - 113913}} \newline \newline
	\textbf{{\large Prova 1} \\
	\vspace{9pt}
	{\large Questão A}} \\
	\noindent\rule{\textwidth}{0.4pt}
	\newline
\end{center}

\textbf{{\large Observações:}}
\begin{itemize}
	\item As provas também serão corrigidas por um \textbf{corretor automático}, portanto é necessário que as entradas e saídas do seu programa estejam conforme o padrão especificado em cada questão (exemplo de entrada e saída). Por exemplo, não use mensagens escritas durante o desenvolvimento do seu código como “Informe a primeira entrada”. Estas mensagens não são tratadas pelo corretor, portanto a correção irá resultar em resposta errada, mesmo que seu código esteja correto.
	\item Serão testadas várias entradas além das que foram dadas como exemplo, assim como as listas.
	\item Assim como as listas, as provas devem ser feitas na versão Python 3 ou superior.
	\item Leia com atenção e faça \textbf{exatamente} o que está sendo pedido.
\end{itemize}
\newpage % Questão A 
\begin{center}
\textbf{{\Large Questão A - Fatorial}}
\end{center}
\vspace{5pt}
Na matemática, o fatorial de um número natural n, representado por $n!$, é o produto de todos os inteiros positivos menores ou iguais a n. A notação $n!$ foi introduzida por Christian Kramp em 1808. \newline \newline Leia uma sequência de inteiros positivos do teclado, um por linha. A sequência termina quando for lido um inteiro menor que 0 (que não fará parte da sequência de números lidos).
\newline \newline
\textbf{{\large Entrada}} \newline
Cada linha de entrada conterá um inteiro \textbf{\textit{k}}, a linha que conter \textbf{k} $<$ \textbf{0} deverá ser desconsiderada. Considere que pelo menos um \textbf{\textit{k}} $\geq$ \textbf{\textit{0}} será lido.
\newline \newline
\textbf{{\large Saída}} \newline
Para cada número \textbf{\textit{k}} $\geq$ \textbf{\textit{0}} lido calcule e imprima o seu fatorial.
\newline
\begin{table}[H]
\centering
\begin{tabular}{|l|l|}
\hline
\textbf{Exemplo de Entrada}                                       & \textbf{Exemplo de Saída}                             \\ \hline
\begin{tabular}[c]{@{}l@{}}0 \\ 4 \\ 3  \\2 \\-1\end{tabular}   & \begin{tabular}[c]{@{}l@{}}1\\ 24\\ 6 \\ 2 \end{tabular} \\ \hline
\begin{tabular}[c]{@{}l@{}}1 \\ 1\\ 5 \\6 \\7 \\-2\end{tabular}  & \begin{tabular}[c]{@{}l@{}}1\\ 1\\ 
120\\720 \\5040\end{tabular} \\ \hline
\begin{tabular}[c]{@{}l@{}}2 \\ 1 \\ -1\end{tabular} & \begin{tabular}[c]{@{}l@{}}2\\ 1\end{tabular} \\ \hline
\end{tabular}
\caption{Questão A}
\end{table}
\flushright
\textbf{\Large Boa Prova!}
\end{document}