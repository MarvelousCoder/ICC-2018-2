\documentclass[a4paper, 12pt]{article}

\usepackage[top=2cm, bottom=2cm, left=2.5cm, right=2.5cm]{geometry}
\usepackage[utf8]{inputenc}
\usepackage{amsmath, amsfonts, amssymb}
\usepackage{graphicx} % inserir figuras - \includegraphics[scale=•]{•}
\usepackage{float} % ignorar regras de tipografia e inserir figura aonde queremos.
\usepackage[brazil]{babel} % Trocar Figure para Figura.
\usepackage{indentfirst}
\pagestyle{empty}


\begin{document}
\begin{figure}[H]
	\includegraphics[scale=0.9]{UnB_CiC_Logo.jpg}
\end{figure}
\noindent\rule{\textwidth}{0.4pt}
\begin{center}
	\textbf{{\Large Introdução à Ciência da Computação - 113913}} \newline \newline
	\textbf{{\large Prova 1} \\
	\vspace{9pt}
	{\large Questão A}} \\
	\noindent\rule{\textwidth}{0.4pt}
	\newline
\end{center}

\textbf{{\large Observações:}}
\begin{itemize}
	\item As provas serão corrigidas por um corretor automático, portanto é necessário que as entradas e saídas do seu programa estejam conforme o padrão especificado em cada questão (exemplo de entrada e saída).
	\item Por exemplo, não use mensagens escritas durante o desenvolvimento do seu código como “Informe a primeira entrada".
	\item Estas mensagens não são tratadas pelo corretor, portanto a correção irá resultar em resposta errada, mesmo que seu código esteja correto.
	\item Serão testadas várias entradas além das que foram dadas como exemplo, assim como as listas.
	\item Assim como as listas, as provas devem ser feitas na versão Python 3 ou superior.
	\item Cada questão (A e B) vale 50\% da nota da Prova 1.
	\item Leia com atenção e faça \textbf{exatamente} o que está sendo pedido.


\end{itemize}
\newpage % Questão A 
\begin{center}
\textbf{{\Large Questão A - Outorgas em bacias hidrográficas}}
\end{center}

\vspace{5pt} 

Para fins de estudos hidrológicos, uma Bacia Hidrográfica (BH) é considerada um conjunto de terras drenadas por um corpo d’agua principal e seus afluentes. Vários autores consideram a importância do uso de conceito de BH similar ao de ecossistema, tanto para estudos como para o Gerenciamento Ambiental. O uso das águas em uma bacia é feito por meio de outorgas para diversas atividades como irrigação, bombeamento, piscicultura, abastecimento e mineração.


Calcule a atividade que mais consome água em uma BH, a que menos consome água, o consumo médio de água e o maior consumidor de água da BH identificado pelo número do processo.
\newline \newline
\textbf{{\large Entrada}} \newline
Um número N que identifica a quantidade de outorgas na BH. \newline
N linhas, cada linha com uma outorga identificada pelo número do processo, a atividade e a vazão em m\textsuperscript{3}/s.
\newline \newline
\textbf{{\large Saída}} \newline
Nome da atividade que mais consome água em uma BH. \newline
Nome da atividade que menos consome água em uma BH. \newline
O consumo médio de água das atividades da BH em m\textsuperscript{3}/s e duas casas decimais de precisão
Número do processo que é o maior consumidor de água.
\newline \newline
\newline
\begin{table}[H]
\centering
\begin{tabular}{|l|l|}
\hline
\textbf{Entrada}                                                                                                              & \textbf{Saída}                                                      \\ \hline
\begin{tabular}[c]{@{}l@{}}3\\ 11/21 IRRIGAÇÃO 40\\ 33/47 BOMBEAMENTO 70\\ 121/342 MINERAÇÃO 22\end{tabular}                                                                                                          & \begin{tabular}[c]{@{}l@{}}BOMBEAMENTO\\ MINERAÇÃO\\ 44.00\\ 33/47\end{tabular}        \\ \hline
\begin{tabular}[c]{@{}l@{}}5\\ 12/21 ABASTECIMENTO 120.2\\ 543/23 BOMBEAMENTO 32\\ 354/234 PISCICULTURA 12\\ 876/12 IRRIGAÇÃO 120\\ 123/234 ABASTECIMENTO 145\end{tabular}                                            & \begin{tabular}[c]{@{}l@{}}ABASTECIMENTO\\ PISCICULTURA\\ 85.84\\ 123/234\end{tabular} \\ \hline
\begin{tabular}[c]{@{}l@{}}7\\ 12/35 IRRIGAÇÃO 30\\ 13/28 BOMBEAMENTO 64\\ 17/68 ABASTECIMENTO 20\\ 548/378 PISCICULTURA 0.38\\ 2123/646 MINERAÇÃO 44.5\\ 545/5465 IRRIGAÇÃO 100\\ 5458/545 IRRIGAÇÃO 80\end{tabular} & \begin{tabular}[c]{@{}l@{}}IRRIGAÇÃO\\ PISCICULTURA\\ 48.41\\ 545/5465\end{tabular}    \\ \hline

\end{tabular}
\caption{Questão A}
\end{table}
\flushright
\textbf{\Large Boa Prova!}
\end{document}