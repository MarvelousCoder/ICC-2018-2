\documentclass[a4paper, 12pt]{article}

\usepackage[top=2cm, bottom=2cm, left=2.5cm, right=2.5cm]{geometry}
\usepackage[utf8]{inputenc}
\usepackage{amsmath, amsfonts, amssymb}
\usepackage{graphicx} % inserir figuras - \includegraphics[scale=•]{•}
\usepackage{float} % ignorar regras de tipografia e inserir figura aonde queremos.
\usepackage[brazil]{babel} % Trocar Figure para Figura.
\usepackage{indentfirst}
\pagestyle{empty}


\begin{document}
\begin{figure}[H]
	\includegraphics[scale=0.9]{UnB_CiC_Logo.jpg}
\end{figure}
\noindent\rule{\textwidth}{0.4pt}
\begin{center}
	\textbf{{\Large Introdução à Ciência da Computação - 113913}} \newline \newline
	\textbf{{\large Prova 2} \\
	\vspace{9pt}
	{\large Questão A}} \\
	\noindent\rule{\textwidth}{0.4pt}
	\newline
\end{center}

\textbf{{\large Observações:}}
\begin{itemize}
	\item As provas também serão corrigidas por um \textbf{corretor automático}, portanto é necessário que as entradas e saídas do seu programa estejam conforme o padrão especificado em cada questão (exemplo de entrada e saída). Por exemplo, não use mensagens escritas durante o desenvolvimento do seu código como “Informe a primeira entrada”. Estas mensagens não são tratadas pelo corretor, portanto a correção irá resultar em resposta errada, mesmo que seu código esteja correto.
	\item Serão testadas várias entradas além das que foram dadas como exemplo, assim como as listas.
	\item Assim como as listas, as provas devem ser feitas na versão Python 3 ou superior.
	\item \textbf{Questão A valerá 40\% da nota da Prova 2 e a Questão B valerá 60\% da nota da Prova 2}.
	\item Leia com atenção e faça \textbf{exatamente} o que está sendo pedido.
\end{itemize}
\newpage % Questão A 
\begin{center}
\textbf{{\Large Questão A - Caviar}}
\end{center}
\vspace{5pt}
Caviar é um alimento requintado feito a partir das ovas do peixe esturjão.
Muitas pessoas gostam bastante de apreciá-lo em torradas ou canapés. Na cidade da Cacóvia, houve um acréscimo na qualidade de vida e, por consequência, na procura por caviar. \newline \newline
Pescadores estão realizando exploração predatória em algumas áreas do
lago adjacente, o que está perturbando a vida dos esturjões de lá,
especialmente do Escamoso, um peixe cheio de personalidade. \newline \newline
Vendo que não teria outra forma, Escamoso recorreu ao programador mais
próximo, você, para resolver o seu problema. Ele quer que você escreva um
programa que, dado as coordenadas que definem um retângulo onde ocorre
a exploração, determine se é seguro ou não ele depositar suas ovas num
local desejado. \newline \newline
\textbf{{\large Entrada}}
\begin{itemize}
\item A primeira linha da entrada contém dois inteiros $X_e$, $Y_e$, as coordenadas do
canto inferior esquerdo do retângulo da exploração;
\item A segunda linha da entrada contém dois inteiros $X_d$, $Y_d$, as coordenadas do
canto superior direito do retângulo da exploração;
\item Por fim, a terceira linha da entrada contém dois inteiros $X_o$, $Y_o$, as
coordenadas em que Escamoso deseja depositar suas ovas.
\end{itemize}
\textbf{{\large Saída}} \newline
Seu programa deve imprimir uma única linha na saída padrão, contendo
``Seguro!'' caso as coordenadas $X_o$, $Y_o$ estejam fora do retângulo da
exploração, e ``Cuidado!'', caso contrário.
\newline
\begin{table}[H]
\centering
\begin{tabular}{|l|l|}
\hline
\textbf{Exemplo de Entrada}                               & \textbf{Exemplo de Saída} \\ \hline
\begin{tabular}[c]{@{}l@{}}2 2\\ 4 4\\ 3 4\end{tabular}   & Cuidado!                  \\ \hline
\begin{tabular}[c]{@{}l@{}}7 5\\ 10 15\\ 1 1\end{tabular} & Seguro!                   \\ \hline
\end{tabular}
\end{table}
\flushright
\textbf{\Large Boa Prova!}
\end{document}