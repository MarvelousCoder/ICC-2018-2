\documentclass[a4paper, 12pt]{article}

\usepackage[top=2cm, bottom=2cm, left=2.5cm, right=2.5cm]{geometry}
\usepackage[utf8]{inputenc}
\usepackage[utf8x]{inputenc}
\usepackage{amsmath, amsfonts, amssymb}
\usepackage{graphicx} % inserir figuras - \includegraphics[scale=•]{•}
\usepackage{float} % ignorar regras de tipografia e inserir figura aonde queremos.
\usepackage[brazil]{babel} % Trocar Figure para Figura.
\usepackage{indentfirst}
\pagestyle{empty}


\begin{document}
\begin{figure}[H]
	\includegraphics[scale=0.9]{UnB_CiC_Logo.jpg}
\end{figure}
\noindent\rule{\textwidth}{0.4pt}
\begin{center}
	\textbf{{\Large Introdução à Ciência da Computação - 113913}} \newline \newline
	\textbf{{\large Prova 2} \\
	\vspace{9pt}
	{\large Questão A}} \\
	\noindent\rule{\textwidth}{0.4pt}
	\newline
\end{center}

\textbf{{\large Observações:}}
\begin{itemize}
	\item As provas também serão corrigidas por um \textbf{corretor automático}, portanto é necessário que as entradas e saídas do seu programa estejam conforme o padrão especificado em cada questão (exemplo de entrada e saída). Por exemplo, não use mensagens escritas durante o desenvolvimento do seu código como “Informe a primeira entrada”. Estas mensagens não são tratadas pelo corretor, portanto a correção irá resultar em resposta errada, mesmo que seu código esteja correto.
	\item Serão testadas várias entradas além das que foram dadas como exemplo, assim como as listas.
	\item Assim como as listas, as provas devem ser feitas na versão Python 3 ou superior.
	\item \textbf{Cada questão (A e B) vale 50\% da nota da prova 2}.
	\item Leia com atenção e faça \textbf{exatamente} o que está sendo pedido.
\end{itemize}
\newpage % Questão A 
\begin{center}
\textbf{{\Large Questão A - Pesquisa de produtos}}
\end{center}
\vspace{5pt}


Um determinado site de venda de produtos utiliza tags para identificação de seus produtos. O CEO da empresa deseja que os desenvolvedores façam um sistema de buscas que ache os produtos pesquisados pelos clientes a partir dessas tags de forma rápida e eficiente.
\newline \newline
\textbf{{\large Entrada}} \newline

A primeira linha representa o inteiro N que diz quantos produtos existem na base de dados.

As N seguintes linhas sao compostas pelo nome do produto e quatro tags T1, T2, T3 e T4 que identificam o produto.

A ultima linha contem a tag pesquisada pelo cliente.
\newline \newline
\textbf{{\large Saída}} \newline

O programa deve procurar as tags pesquisadas e imprimir todos os produtos que possuem essa tag em linhas separadas e na ordem que eles apareceram no input.
\\
\newline \newline
\textbf{{\large Exemplo}} \newline

\begin{table}[htb]
\centering
\begin{tabular}{|l|l|l}
\cline{1-2}
Entrada                                                                                                                                                                                        & Saida                                                            &  \\ \cline{1-2}
\begin{tabular}[c]{@{}l@{}}3\\ cabide roupas armario organizacao guarda-roupas\\ vassoura limpeza poeira cabo poeira\\ humidificador seco humidificar temperatura poeira\\ poeira\end{tabular} & \begin{tabular}[c]{@{}l@{}}vassoura\\ humidificador\end{tabular} &  \\ \cline{1-2}
\begin{tabular}[c]{@{}l@{}}2\\ sofa conforto sentar sala espaco\\ celular contatos ligar telefone zap\\ sala\end{tabular}                                                                      & sala                                                             &  \\ \cline{1-2}
\end{tabular}
\end{table}


\flushright
\textbf{\Large Boa Prova!}
\end{document}