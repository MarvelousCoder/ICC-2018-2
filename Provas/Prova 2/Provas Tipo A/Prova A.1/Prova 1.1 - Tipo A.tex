\documentclass[a4paper, 12pt]{article}

\usepackage[top=2cm, bottom=2cm, left=2.5cm, right=2.5cm]{geometry}
\usepackage[utf8]{inputenc}
\usepackage{amsmath, amsfonts, amssymb}
\usepackage{graphicx} % inserir figuras - \includegraphics[scale=•]{•}
\usepackage{float} % ignorar regras de tipografia e inserir figura aonde queremos.
\usepackage[brazil]{babel} % Trocar Figure para Figura.
\usepackage{indentfirst}
\pagestyle{empty}


\begin{document}
\begin{figure}[H]
	\includegraphics[scale=0.9]{UnB_CiC_Logo.jpg}
\end{figure}
\noindent\rule{\textwidth}{0.4pt}
\begin{center}
	\textbf{{\Large Introdução à Ciência da Computação - 113913}} \newline \newline
	\textbf{{\large Prova 2} \\
	\vspace{9pt}
	{\large Questão A}} \\
	\noindent\rule{\textwidth}{0.4pt}
	\newline
\end{center}

\textbf{{\large Observações:}}
\begin{itemize}
	\item As provas também serão corrigidas por um \textbf{corretor automático}, portanto é necessário que as entradas e saídas do seu programa estejam conforme o padrão especificado em cada questão (exemplo de entrada e saída). Por exemplo, não use mensagens escritas durante o desenvolvimento do seu código como “Informe a primeira entrada”. Estas mensagens não são tratadas pelo corretor, portanto a correção irá resultar em resposta errada, mesmo que seu código esteja correto.
	\item Serão testadas várias entradas além das que foram dadas como exemplo, assim como as listas.
	\item Assim como as listas, as provas devem ser feitas na versão Python 3 ou superior.
	\item \textbf{Questão A valerá 40\% da nota da Prova 2 e a Questão B valerá 60\% da nota da Prova 2}.
	\item Leia com atenção e faça \textbf{exatamente} o que está sendo pedido.
\end{itemize}
\newpage % Questão A 
\begin{center}
\textbf{{\Large Questão A - Charmander e Pikachu}}
\end{center}
\vspace{5pt}
Charmander está jogando um jogo com o Pikachu feito para Pokémons. As regras são simples. Os pokémons têm que dizer palavras em turnos. Não é possível dizer uma palavra que já foi dita. Charmander começa. O jogador que não puder dizer uma nova palavra perde. \newline \newline
É dados duas listas de palavras familiares ao Charmander e ao Pikachu. Você pode determinar quem vence o jogo, sabendo que os dois jogam de forma otimizada?
\newline \newline
\textbf{{\large Entrada}} \newline
A primeira linha contém dois inteiros \textbf{n} e \textbf{m} $\left( 1\leq\textrm{n, m} \leq 1000 \right)$ – o número de palavras que Charmander e Pikachu conhecem, respectivamente. \newline
Em seguida temos n strings, uma por linha – as palavras familiares para o Charmander. \newline
Depois temos m strings, uma por linha – as palavras familiares para o Pikachu. \newline
Um Pokémon não pode conhecer uma palavra mais de uma vez (as strings são únicas), mas algumas palavras podem ser conhecidas pelos dois pokémons. Cada palavra é não vazia.
\newline \newline
\textbf{{\large Saída}} \newline
Na primeira linha imprima a resposta – ``SIM'' se o Charmander vencer e ``NAO'' caso contrário. Lembrando que os dois jogam de forma otimizada. Na próxima linha imprima na tela a primeira palavra da entrada.
\newline \newline
\textbf{{\large Dica}} \newline
Considere primeiramente o número de palavras que são conhecidas simultaneamente pelos dois jogadores e verifique o que acontece.
\newline \newline
\textbf{{\large Nota}} \newline
No primeiro exemplo o Charmander conhece mais palavras e ganha sem esforço. \newline
No segundo exemplo, se o Charmander dizer ``chaa'' primeiro, então o Pikachu não pode usar essa palavra mais. Pikachu só poderá dizer ``chuu'', então o Charmander fala ``derr'' e vence.
\newline
\begin{table}[H]
\centering
\begin{tabular}{|l|l|}
\hline
\textbf{Exemplo de Entrada}                                                     & \textbf{Exemplo de Saída}                                \\ \hline
\begin{tabular}[c]{@{}l@{}}2 1\\ charrr\\ manderr\\ pikaa\end{tabular}          & \begin{tabular}[c]{@{}l@{}}SIM\\ charrr\end{tabular}     \\ \hline
\begin{tabular}[c]{@{}l@{}}2 2\\ chaa\\ derr\\ chaa\\ chuu\end{tabular}         & \begin{tabular}[c]{@{}l@{}}SIM\\ chaa\end{tabular}       \\ \hline
\begin{tabular}[c]{@{}l@{}}1 2\\ charmander\\ charmander\\ pikachu\end{tabular} & \begin{tabular}[c]{@{}l@{}}NAO\\ charmander\end{tabular} \\ \hline
\end{tabular}
\end{table}
\flushright
\textbf{\Large Boa Prova!}
\end{document}