\documentclass[a4paper, 12pt]{article}

\usepackage[top=2cm, bottom=2cm, left=2.5cm, right=2.5cm]{geometry}
\usepackage[utf8]{inputenc}
\usepackage{amsmath, amsfonts, amssymb}
\usepackage{graphicx} % inserir figuras - \includegraphics[scale=•]{•}
\usepackage{float} % ignorar regras de tipografia e inserir figura aonde queremos.
\usepackage[brazil]{babel} % Trocar Figure para Figura.
\usepackage{indentfirst}
\pagestyle{empty}


\begin{document}
\begin{figure}[H]
	\includegraphics[scale=0.9]{UnB_CiC_Logo.jpg}
\end{figure}
\noindent\rule{\textwidth}{0.4pt}
\begin{center}
	\textbf{{\Large Introdução à Ciência da Computação - 113913}} \newline \newline
	\textbf{{\large Prova 2} \\
	\vspace{9pt}
	{\large Questão A}} \\
	\noindent\rule{\textwidth}{0.4pt}
	\newline
\end{center}

\textbf{{\large Observações:}}
\begin{itemize}
	\item As provas também serão corrigidas por um \textbf{corretor automático}, portanto é necessário que as entradas e saídas do seu programa estejam conforme o padrão especificado em cada questão (exemplo de entrada e saída). Por exemplo, não use mensagens escritas durante o desenvolvimento do seu código como “Informe a primeira entrada”. Estas mensagens não são tratadas pelo corretor, portanto a correção irá resultar em resposta errada, mesmo que seu código esteja correto.
	\item Serão testadas várias entradas além das que foram dadas como exemplo, assim como as listas.
	\item Assim como as listas, as provas devem ser feitas na versão Python 3 ou superior.
	\item \textbf{Questão A valerá 40\% da nota da Prova 2 e a Questão B valerá 60\% da nota da Prova 2}.
	\item Leia com atenção e faça \textbf{exatamente} o que está sendo pedido.
\end{itemize}
\newpage % Questão A 
\begin{center}
\textbf{{\Large Questão A - Arranjos}}
\end{center}
\vspace{5pt}
Albertíneo é um jovem que tem uma compulsão por trocar a ordem de tudo
que vê. Potes, copos, panelas e até seus próprios amiguinhos. \newline \newline
Seus amigos, às vezes, se irritam bastante pois quando ele entra nos seus
quartos, sai tudo trocado. Até que um dia, um deles ouviu falar que
programação é a solução de todos os problemas e procurou um habilidoso
programador python para resolver a situação, você. \newline \newline
Sejam as seguintes permutas:
\begin{enumerate}
\item A, B, C $\longrightarrow$ C, B, A
\item A, B, C $\longrightarrow$ A, C, B
\item A, B, C $\longrightarrow$ B, A, C
\end{enumerate}
Seu trabalho é, dada a ordem de execução de cada uma das permutas 1, 2 ou 3, determinar a configuração final dos objetos.
\newline \newline
\textbf{{\large Entrada}} \newline
A primeira linha do programa contém um inteiro \textbf{N}, o número de permutas
executadas. As próximas \textbf{N} linhas contêm, cada uma, um inteiro (1, 2, ou 3), o
identificador da permuta, como descrito acima.
\newline \newline
\textbf{{\large Saída}} \newline
Seu programa deve imprimir A, B, C, na tela, permutados conforme Albertíneo
reorganizou-os após deixar a cena do crime.
\newline
\begin{table}[H]
\centering
\begin{tabular}{|l|l|}
\hline
\textbf{Exemplo de Entrada}                           & \textbf{Exemplo de Saída} \\ \hline
\begin{tabular}[c]{@{}l@{}}3\\ 2\\ 1\\ 3\end{tabular} & C B A                     \\ \hline
\begin{tabular}[c]{@{}l@{}}2\\ 1\\ 1\end{tabular}     & A B C                     \\ \hline
\end{tabular}
\end{table}
\flushright
\textbf{\Large Boa Prova!}
\end{document}