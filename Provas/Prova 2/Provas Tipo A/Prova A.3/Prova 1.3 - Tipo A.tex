\documentclass[a4paper, 12pt]{article}

\usepackage[top=2cm, bottom=2cm, left=2.5cm, right=2.5cm]{geometry}
\usepackage[utf8]{inputenc}
\usepackage{amsmath, amsfonts, amssymb}
\usepackage{graphicx} % inserir figuras - \includegraphics[scale=•]{•}
\usepackage{float} % ignorar regras de tipografia e inserir figura aonde queremos.
\usepackage[brazil]{babel} % Trocar Figure para Figura.
\usepackage{indentfirst}
\pagestyle{empty}


\begin{document}
\begin{figure}[H]
	\includegraphics[scale=0.9]{UnB_CiC_Logo.jpg}
\end{figure}
\noindent\rule{\textwidth}{0.4pt}
\begin{center}
	\textbf{{\Large Introdução à Ciência da Computação - 113913}} \newline \newline
	\textbf{{\large Prova 2} \\
	\vspace{9pt}
	{\large Questão A}} \\
	\noindent\rule{\textwidth}{0.4pt}
	\newline
\end{center}

\textbf{{\large Observações:}}
\begin{itemize}
	\item As provas também serão corrigidas por um \textbf{corretor automático}, portanto é necessário que as entradas e saídas do seu programa estejam conforme o padrão especificado em cada questão (exemplo de entrada e saída). Por exemplo, não use mensagens escritas durante o desenvolvimento do seu código como “Informe a primeira entrada”. Estas mensagens não são tratadas pelo corretor, portanto a correção irá resultar em resposta errada, mesmo que seu código esteja correto.
	\item Serão testadas várias entradas além das que foram dadas como exemplo, assim como as listas.
	\item Assim como as listas, as provas devem ser feitas na versão Python 3 ou superior.
	\item \textbf{Questão A valerá 40\% da nota da Prova 2 e a Questão B valerá 60\% da nota da Prova 2}.
	\item Leia com atenção e faça \textbf{exatamente} o que está sendo pedido.
\end{itemize}
\newpage % Questão A 
\begin{center}
\textbf{{\Large Questão A - Antônimos}}
\end{center}
\vspace{5pt}
Antônima é uma jovem que tem um disfunção cerebral sem cura. Sempre que ela
tenta falar um nome (substantivo ou adjetivo), ela fala o antônimo. Por exemplo, se
ela quisesse elogiar a Ana com ``Você é linda'', ela diria ``Você é horrível''. \newline \newline
Seus amigos, cansados de tentar entender se ela estava os ofendendo ou
demonstrando carinho, procuraram um programador para resolver o problema
deles. Não demorou muito, e encontraram você. Seu trabalho é criar um programa que,
fornecido um dicionário de antônimos e uma frase de Antônima, deve decifrar o
que ela quis dizer.
\newline \newline
\textbf{{\large Entrada}} \newline
A primeira linha da entrada consiste de um inteiro \textbf{N}, o número de entradas do
dicionário fornecido. \newline
As próximas \textbf{N} linhas contêm, cada uma, duas strings \textbf{O} e \textbf{A}, a palavra de Origem
e seu Antônimo, respectivamente. \newline
A última linha da entrada, contém, enfim, a frase de Antônima. \newline
Considere que pontuação sempre estará separada por espaços
\newline \newline
\textbf{{\large Saída}} \newline
Seu programa deve imprimir uma única linha, a frase ``traduzida''.
\newline
\begin{table}[H]
\centering
\begin{tabular}{|l|l|}
\hline
\textbf{Exemplo de Entrada}                                                                                                                                                                             & \textbf{Exemplo de Saída}                                                                                                   \\ \hline
\begin{tabular}[c]{@{}l@{}}2\\ feio bonito\\ grande pequeno\\ você é grande e muito feio\end{tabular}                                                                                                   & você é pequeno e muito bonito                                                                                               \\ \hline
\begin{tabular}[c]{@{}l@{}}3\\ cheiroso fedido\\ competente incompetente\\ comprometido relaxado\\ eu acho que você é um grande de um \\ competente comprometido , seu\\ cheiroso de meia!\end{tabular} & \begin{tabular}[c]{@{}l@{}}eu acho que você é um grande de um\\ incompetente relaxado , seu fedido de\\ meia !\end{tabular} \\ \hline
\end{tabular}
\end{table}
\flushright
\textbf{\Large Boa Prova!}
\end{document}