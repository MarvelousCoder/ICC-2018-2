\documentclass[a4paper, 12pt]{article}

\usepackage[top=2cm, bottom=2cm, left=2.5cm, right=2.5cm]{geometry}
\usepackage[utf8]{inputenc}
\usepackage[utf8x]{inputenc}
\usepackage{amsmath, amsfonts, amssymb}
\usepackage{graphicx} % inserir figuras - \includegraphics[scale=•]{•}
\usepackage{float} % ignorar regras de tipografia e inserir figura aonde queremos.
\usepackage[brazil]{babel} % Trocar Figure para Figura.
\usepackage{indentfirst}
\pagestyle{empty}


\begin{document}
\begin{figure}[H]
	\includegraphics[scale=0.9]{UnB_CiC_Logo.jpg}
\end{figure}
\noindent\rule{\textwidth}{0.4pt}
\begin{center}
	\textbf{{\Large Introdução à Ciência da Computação - 113913}} \newline \newline
	\textbf{{\large Prova 2} \\
	\vspace{9pt}
	{\large Questão A}} \\
	\noindent\rule{\textwidth}{0.4pt}
	\newline
\end{center}

\textbf{{\large Observações:}}
\begin{itemize}
	\item As provas também serão corrigidas por um \textbf{corretor automático}, portanto é necessário que as entradas e saídas do seu programa estejam conforme o padrão especificado em cada questão (exemplo de entrada e saída). Por exemplo, não use mensagens escritas durante o desenvolvimento do seu código como “Informe a primeira entrada”. Estas mensagens não são tratadas pelo corretor, portanto a correção irá resultar em resposta errada, mesmo que seu código esteja correto.
	\item Serão testadas várias entradas além das que foram dadas como exemplo, assim como as listas.
	\item Assim como as listas, as provas devem ser feitas na versão Python 3 ou superior.
	\item \textbf{Cada questão (A e B) vale 50\% da nota da prova 2}.
	\item Leia com atenção e faça \textbf{exatamente} o que está sendo pedido.
\end{itemize}
\newpage % Questão A 
\begin{center}
\textbf{{\Large Questão A - Restaurante}}
\end{center}
\vspace{5pt}

Renato é dono de um retaurante e decidiu fazer um acordo com a empresa X que fica ao lado. Cada um dos trabalhadores da empresa X pode escolher um prato para o almoço sendo que o cliente pode mudar o pedido até meio dia. Renato decidiu contratar um programador para ajuda-lo nessa tarefa.
\newline \newline
\textbf{{\large Entrada}} \newline

A primeira linha consiste de um inteiro N que é o número total de pedidos realizados.
Cada um dos pedidos consiste em uma string sem espaços S que representa o nome do cliente e uma string que pode conter espaços D que é o nome do prato pedido.
Se um mesmo cliente pedir mais de um prato (mudar o pedido), apenas o último deve ser levado em consideração.
\newline \newline
\textbf{{\large Saída}} \newline

A primeira linha da saída deve ser o número de clientes de fato atendidos J. Em seguida, as próximas J linhas devem conter o nome dos pratos pedidos ordenados em ordem alfabética do nome do cliente.
\\
\newline \newline
\textbf{{\large Exemplo}} \newline

\begin{table}[htb]
\centering
\begin{tabular}{|l|l|l}
\cline{1-2}
Entrada                                                                                                                         & Saida                                                                                       &  \\ \cline{1-2}
\begin{tabular}[c]{@{}l@{}}3\\ João Bife do oião\\ Ricardo Salsicha recheada com queijo\\ João Cuzcuz com leite\end{tabular} & \begin{tabular}[c]{@{}l@{}}2\\ Cuzcuz com leite\\ Salsicha recheada com queijo\end{tabular} &  \\ \cline{1-2}
\begin{tabular}[c]{@{}l@{}}2\\ Zé abacates\\ Alberto sashimi à passarinho\end{tabular}                                        & \begin{tabular}[c]{@{}l@{}}2\\ sashimi à passarinho\\ abacates\end{tabular}                &  \\ \cline{1-2}
\end{tabular}
\end{table}

\flushright
\textbf{\Large Boa Prova!}
\end{document}