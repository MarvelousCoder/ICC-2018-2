\documentclass[a4paper, 12pt]{article}

\usepackage[top=2cm, bottom=2cm, left=2.5cm, right=2.5cm]{geometry}
\usepackage[utf8]{inputenc}
\usepackage{amsmath, amsfonts, amssymb}
\usepackage{graphicx} % inserir figuras - \includegraphics[scale=•]{•}
\usepackage{float} % ignorar regras de tipografia e inserir figura aonde queremos.
\usepackage[brazil]{babel} % Trocar Figure para Figura.
\usepackage{indentfirst}
\pagestyle{empty}


\begin{document}
\begin{figure}[H]
	\includegraphics[scale=0.9]{UnB_CiC_Logo.jpg}
\end{figure}
\noindent\rule{\textwidth}{0.4pt}
\begin{center}
	\textbf{{\Large Introdução à Ciência da Computação - 113913}} \newline \newline
	\textbf{{\large Prova 2} \\
	\vspace{9pt}
	{\large Questão A}} \\
	\noindent\rule{\textwidth}{0.4pt}
	\newline
\end{center}

\textbf{{\large Observações:}}
\begin{itemize}
	\item As provas também serão corrigidas por um \textbf{corretor automático}, portanto é necessário que as entradas e saídas do seu programa estejam conforme o padrão especificado em cada questão (exemplo de entrada e saída). Por exemplo, não use mensagens escritas durante o desenvolvimento do seu código como “Informe a primeira entrada”. Estas mensagens não são tratadas pelo corretor, portanto a correção irá resultar em resposta errada, mesmo que seu código esteja correto.
	\item Serão testadas várias entradas além das que foram dadas como exemplo, assim como as listas.
	\item Assim como as listas, as provas devem ser feitas na versão Python 3 ou superior.
	\item \textbf{Questão A valerá 40\% da nota da Prova 2 e a Questão B valerá 60\% da nota da Prova 2}.
	\item Leia com atenção e faça \textbf{exatamente} o que está sendo pedido.
\end{itemize}
\newpage % Questão A 
\begin{center}
\textbf{{\Large Questão A - Carros}}
\end{center}
\vspace{5pt}
A1A é um lava-jato na região de Albuquerque que recebe um grande volume de
carros todos os dias. Seu dono, Bogdan Wolynetz, tem tido problemas recentes
com alguns de seus funcionários e precisa de alguma forma de gerenciar os carros
sujos e necessitados de cera de seus clientes. Para isso, ele decidiu contratar um desenvolvedor python, você, para criar um sistema de gerenciamento sólido para seu estabelecimento. \newline \newline
\textbf{{\large Entrada}} \newline
Cada linha da entrada começa com uma letra `\textbf{E}', `\textbf{R}' ou `\textbf{F}', indicando o tipo de
transação a ser realizada.
Se a transação for do tipo `\textbf{E}' (entrar), o resto da linha conterá o primeiro nome do
cliente (sem espaços), e qual o tipo de serviço a ser realizado (`cera' ou `lavagem').
Caso a transação for `\textbf{R}' (retirar), o resto da linha conterá somente o primeiro nome
do cliente a ter o carro retirado.
Por fim, a última linha da entrada será sempre do tipo `\textbf{F}', indicando o fechamento
do lava-jato. \newline \newline
\textbf{{\large Saída}} \newline
Para cada transação do tipo `\textbf{R}', seu programa deve imprimir qual o último serviço
que foi realizado no carro em questão. Caso não houver nenhum carro cadastrado
sob o nome requisitado, imprimir `Usuário não cadastrado'.
\newline
\begin{table}[H]
\centering
\begin{tabular}{|l|l|}
\hline
\textbf{Exemplo de Entrada}                                                                                                          & \textbf{Exemplo de Saída}                                                                             \\ \hline
\begin{tabular}[c]{@{}l@{}}E Walter Cera\\ R Heisenberg\\ R Walter\\ F\end{tabular}                                                  & \begin{tabular}[c]{@{}l@{}}Usuário não cadastrado\\ cera\end{tabular}                                 \\ \hline
\begin{tabular}[c]{@{}l@{}}R Roberto\\ R André\\ E André lavagem\\ E Rafael cera\\ E André cera\\ R André\\ R André\\ F\end{tabular} & \begin{tabular}[c]{@{}l@{}}Usuário não cadastrado\\ Usuário não cadastrado\\ cera\\ cera\end{tabular} \\ \hline
\end{tabular}
\end{table}
\flushright
\textbf{\Large Boa Prova!}
\end{document}