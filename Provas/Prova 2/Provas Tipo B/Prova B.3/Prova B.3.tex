\documentclass[a4paper, 12pt]{article}

\usepackage[top=2cm, bottom=2cm, left=2.5cm, right=2.5cm]{geometry}
\usepackage[utf8]{inputenc}
\usepackage{amsmath, amsfonts, amssymb}
\usepackage{graphicx} % inserir figuras - \includegraphics[scale=•]{•}
\usepackage{float} % ignorar regras de tipografia e inserir figura aonde queremos.
\usepackage[brazil]{babel} % Trocar Figure para Figura.
\usepackage{indentfirst}
\pagestyle{empty}


\begin{document}
\begin{figure}[H]
	\includegraphics[scale=0.9]{UnB_CiC_Logo.jpg}
\end{figure}
\noindent\rule{\textwidth}{0.4pt}
\begin{center}
	\textbf{{\Large Introdução à Ciência da Computação - 113913}} \newline \newline
	\textbf{{\large Prova 2} \\
	\vspace{9pt}
	{\large Questão B}} \\
	\noindent\rule{\textwidth}{0.4pt}
	\newline
\end{center}

\textbf{{\large Observações:}}
\begin{itemize}
	\item As provas também serão corrigidas por um \textbf{corretor automático}, portanto é necessário que as entradas e saídas do seu programa estejam conforme o padrão especificado em cada questão (exemplo de entrada e saída). Por exemplo, não use mensagens escritas durante o desenvolvimento do seu código como “Informe a primeira entrada”. Estas mensagens não são tratadas pelo corretor, portanto a correção irá resultar em resposta errada, mesmo que seu código esteja correto.
	\item Serão testadas várias entradas além das que foram dadas como exemplo, assim como as listas.
	\item Assim como as listas, as provas devem ser feitas na versão Python 3 ou superior.
	\item \textbf{Questão A valerá 40\% da nota da Prova 2 e a Questão B valerá 60\% da nota da Prova 2}.
	\item Leia com atenção e faça \textbf{exatamente} o que está sendo pedido.
\end{itemize}
\newpage % Questão A 
\begin{center}
\textbf{{\Large Questão B - Cartésia}}
\end{center}
\vspace{5pt}
Você estava andando pela sua já conhecida Cartésia, a cidade em que todas as
quadras são identificadas por coordenadas em um plano cartesiano, quando se
deparou com a sua velha amiga Cunegonde. \newline
Ela aparentava estar muito perdida, e se sentiu muito aliviada ao ver o melhor
programador que conhecia ali, bem na sua frente. Estava salva, enfim. \newline \newline
Cunegonde tinha várias festas para ir e não sabia direito como chegar nelas. Você,
como ótimo programador e altruísta que é, logo se disponibilizou para auxiliá-la a
chegar nas festas que queria. \newline \newline
\textbf{{\large Entrada}} \newline
A primeira linha da entrada consiste de um inteiro \textbf{N}, o número de festas a ser
registradas no sistema. \newline
As próximas \textbf{N} linhas contém, cada uma, uma string sem espaços \textbf{S} e quatro
inteiros $\textbf{X}_o$, $\textbf{Y}_o$, $\textbf{X}_f$, $\textbf{Y}_f$, o nome da festa e as coordenadas da quadra onde
Cunegonde se encontra e as coordenadas da quadra da festa, respectivamente. \newline
A última linha da entrada contém, por fim, uma sequência de strings \textbf{F}, os
identificadores das festas que Cunegonde quer ir hoje. \newline
Considere que duas festas diferentes nunca terão o mesmo identificador. \newline \newline
\textbf{{\large Saída}} \newline
Seu programa deve imprimir múltiplas linhas, uma para cada string \textbf{F} fornecida na
entrada, na ordem de input. Cada linha deve conter dois inteiros $\textbf{D}_x$ e $\textbf{D}_y$, quantas
quadras Cunegonde deverá andar para o Leste, e quantas quadras ela deverá
andar para o Norte, use valores negativos caso ela tenha que andar para o Oeste
ou para o Sul. \newline \newline
É sabido que em Cartésia as coordenadas crescem no sentido Leste e Norte, ou
seja, a posição (3, 0) fica mais à Leste do que (2, 0).
\newline
\begin{table}[H]
\centering
\begin{tabular}{|l|l|}
\hline
\textbf{Exemplo de Entrada}                                                                                                                                                & \textbf{Exemplo de Saída}                                 \\ \hline
\begin{tabular}[c]{@{}l@{}}3\\ festa\_da\_carlinha 0 0 2 2\\ carnaval 2 2 3 3\\ são\_joão\_das\_sisters 2 3 0 5\\ festa\_da\_carlinha são\_joão\_das\_sisters\end{tabular} & \begin{tabular}[c]{@{}l@{}}2 2\\ -2 2\end{tabular}        \\ \hline
\begin{tabular}[c]{@{}l@{}}4\\ natal 4 4 0 0\\ ano-novo 3 8 9 4\\ arquitetura 2 8 9 1\\ halloween 0 0 0 0\\ halloween ano-novo arquitetura\end{tabular}                    & \begin{tabular}[c]{@{}l@{}}0 0\\ 6 -4\\ 7 -7\end{tabular} \\ \hline
\end{tabular}
\end{table}
\flushright
\textbf{\Large Boa Prova!}
\end{document}