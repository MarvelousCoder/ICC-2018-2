\documentclass[a4paper, 12pt]{article}

\usepackage[top=2cm, bottom=2cm, left=2.5cm, right=2.5cm]{geometry}
\usepackage[utf8]{inputenc}
\usepackage{amsmath, amsfonts, amssymb}
\usepackage{graphicx} % inserir figuras - \includegraphics[scale=•]{•}
\usepackage{float} % ignorar regras de tipografia e inserir figura aonde queremos.
\usepackage[brazil]{babel} % Trocar Figure para Figura.
\usepackage{indentfirst}
\pagestyle{empty}


\begin{document}
\begin{figure}[H]
	\includegraphics[scale=0.9]{UnB_CiC_Logo.jpg}
\end{figure}
\noindent\rule{\textwidth}{0.4pt}
\begin{center}
	\textbf{{\Large Introdução à Ciência da Computação - 113913}} \newline \newline
	\textbf{{\large Prova 2} \\
	\vspace{9pt}
	{\large Questão B}} \\
	\noindent\rule{\textwidth}{0.4pt}
	\newline
\end{center}

\textbf{{\large Observações:}}
\begin{itemize}
	\item As provas também serão corrigidas por um \textbf{corretor automático}, portanto é necessário que as entradas e saídas do seu programa estejam conforme o padrão especificado em cada questão (exemplo de entrada e saída). Por exemplo, não use mensagens escritas durante o desenvolvimento do seu código como “Informe a primeira entrada”. Estas mensagens não são tratadas pelo corretor, portanto a correção irá resultar em resposta errada, mesmo que seu código esteja correto.
	\item Serão testadas várias entradas além das que foram dadas como exemplo, assim como as listas.
	\item Assim como as listas, as provas devem ser feitas na versão Python 3 ou superior.
	\item \textbf{Cada questão (A e B) vale 50\% da nota da prova 2}.
	\item Leia com atenção e faça \textbf{exatamente} o que está sendo pedido.
\end{itemize}
\newpage % Questão A 
\begin{center}
\textbf{{\Large Questão B - Algoritmos de Compactação}}
\end{center}
\vspace{5pt}

Uma imagem possui muita redundância espacial, inclusive imagens feitas com caracteres da tabela ASCII. Para diminuir o tamanho dos arquivos com estas imagens, alguém inventou a seguinte codificação. Quando um caracter aparecer de forma seguida (na mesma linha) na imagem, ele será substituído por um número que será a quantidade de vezes que o caractere aparece seguido na imagem mais o caracter propriamente dito.

\newline \newline
A sequência "3@5!2)4A" por exemplo, se torna "@@@!!!!!))AAAA".
\\

\newline \newline
\textbf{{\large Entrada}} \newline

A imagem compactada, linha por linha. A imagem termina quando a palavra “fim” aparece isolada em uma linha (última linha).
\\

\newline \newline
\textbf{{\large Saída}} \newline

Imagem em caracteres ASCII. Ao final, o tamanho original da imagem e o seu tamanho após a compactação.
\newline
\\
\\
\newline \newline

\begin{table}[H]
\centering
\begin{tabular}{|l|l|}
\hline
Entrada                                                                                                            & Saida                                                                                                      \\ \hline
\begin{tabular}[c]{@{}l@{}}3\#1O3\#\\ 2\#1O1\#1O2\#\\ 1\#1O3O1O1\#\\ 2\#1O1\#1O2\#\\ 3\#1O3\#\\ fim\end{tabular} & \begin{tabular}[c]{@{}l@{}}\#\#\#O\#\#\#\\ \#\#O\#O\#\#\\ \#O\#\#\#O\#\\ \#\#O\#O\#\#\\ \#\#\#O\#\#\#\end{tabular}  \\ \hline
\begin{tabular}[c]{@{}l@{}}6O\\ 6A\\ 6C\\ 6D\\ 1A1E1A1E1A1E\\fim\end{tabular}   
& \begin{tabular}[c]{@{}l@{}}OOOOOO\\ AAAAAA\\ CCCCCC\\ DDDDDD\\ AEAEAE\end{tabular}                                                                \\ \hline
\begin{tabular}[c]{@{}l@{}}3(1O3)\\ 2(2O2)\\ 1(3O1)\\ 4O\\fim\end{tabular}   
&
\begin{tabular}[c]{@{}l@{}}(((O)))\\ ((OO))\\ (OOO)\\ OOOO\end{tabular}                                                                              \\ \hline
\end{tabular}
\end{table}

\flushright
\textbf{\Large Boa Prova!}
\end{document}