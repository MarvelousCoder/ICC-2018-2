\documentclass[a4paper, 12pt]{article}

\usepackage[top=2cm, bottom=2cm, left=2.5cm, right=2.5cm]{geometry}
\usepackage[utf8]{inputenc}
\usepackage{amsmath, amsfonts, amssymb}
\usepackage{graphicx} % inserir figuras - \includegraphics[scale=•]{•}
\usepackage{float} % ignorar regras de tipografia e inserir figura aonde queremos.
\usepackage[brazil]{babel} % Trocar Figure para Figura.
\usepackage{indentfirst}
\pagestyle{empty}


\begin{document}
\begin{figure}[H]
	\includegraphics[scale=0.9]{UnB_CiC_Logo.jpg}
\end{figure}
\noindent\rule{\textwidth}{0.4pt}
\begin{center}
	\textbf{{\Large Introdução à Ciência da Computação - 113913}} \newline \newline
	\textbf{{\large Prova 2} \\
	\vspace{9pt}
	{\large Questão B}} \\
	\noindent\rule{\textwidth}{0.4pt}
	\newline
\end{center}

\textbf{{\large Observações:}}
\begin{itemize}
	\item As provas também serão corrigidas por um \textbf{corretor automático}, portanto é necessário que as entradas e saídas do seu programa estejam conforme o padrão especificado em cada questão (exemplo de entrada e saída). \item Por exemplo, não use mensagens escritas durante o desenvolvimento do seu código como “Informe a primeira entrada”.
	\item Estas mensagens não são tratadas pelo corretor, portanto a correção irá resultar em resposta errada, mesmo que seu código esteja correto.
	\item Serão testadas várias entradas além das que foram dadas como exemplo, assim como as listas.
	\item Assim como as listas, as provas devem ser feitas na versão Python 3 ou superior.
	\item Cada questão (A e B) vale 50\% da nota da Prova 2.
	\item Leia com atenção e faça \textbf{exatamente} o que está sendo pedido.
\end{itemize}
\newpage % Questão A 
\begin{center}
\textbf{{\Large Questão B - Código Binário Alienígena Big Endian}}
\end{center}
\vspace{5pt}
Um povo alienígena deixou rastros de sua escrita na terra. Em uma pedra encontrou os seguintes símbolos a seguir transcritos: “+#+##### *****”. Um pesquisador de vida extraterrestre concluiu que a sequência de símbolos da esquerda é um código binário e a sequência de símbolos da direita é um código unário e que os dois códigos expressam um mesmo valor.
\newline Faça um programa que processe este código e traduza qualquer inteiro base 10 em código binário alienígena.
\newline \newline
\textbf{{\large Entrada}} \newline
Uma linha com uma palavra em código binário alienígena seguida por outra palavra em código unário alienígena, seguida por N linhas. 
\newline Cada linha contém um número inteiro na base 10. N é dado pelo código binário alienígena da primeira linha.
\newline \newline
\textbf{{\large Saída}} \newline
Um número em binário alienígena com M bits, onde M é o número de bist do número alienígida lido na primeira linha.
\newline \newline
\textbf{{\large Dica}} \newline
O bit mais significativo pode estar mais à direita ou mais à esquerda dependendo do povo alienígena.
\newline
\begin{table}[H]
\centering
\begin{tabular}{|l|l|}
\hline
\textbf{Exemplo de Entrada}                                                                          & \textbf{Exemplo de Saída}                                                                                                                     \\ \hline
\begin{tabular}[c]{@{}l@{}}\#\#\#\#\#+\#+ *****\\ 1\\ 2\\ 3\\ 4\\ 5\end{tabular}                  & \begin{tabular}[c]{@{}l@{}}\#\#\#\#\#\#\#+\\ \#\#\#\#\#\#+\#\\ \#\#\#\#\#\#++\\ \#\#\#\#\#+\#\#\\ \#\#\#\#\#+\#+\end{tabular}                 \\ \hline
\begin{tabular}[c]{@{}l@{}}.,. \#\#\\ 1\\ 1\end{tabular}                                             & \begin{tabular}[c]{@{}l@{}}..,\\ ..,\end{tabular}                                                                                             \\ \hline
\begin{tabular}[c]{@{}l@{}}\textless{}\textless{}\textgreater{}\textless ˆˆ\\ 5\\ 12\end{tabular} & \begin{tabular}[c]{@{}l@{}}\textless{}\textgreater{}\textless{}\textgreater\\ \textgreater{}\textless{}\textgreater{}\textless{}\end{tabular} \\ \hline

\end{tabular}
\end{table}
\flushright
\textbf{\Large Boa Prova!}
\end{document}