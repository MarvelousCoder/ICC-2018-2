\documentclass[a4paper, 12pt]{article}

\usepackage[top=2cm, bottom=2cm, left=2.5cm, right=2.5cm]{geometry}
\usepackage[utf8]{inputenc}
\usepackage{amsmath, amsfonts, amssymb}
\usepackage{graphicx} % inserir figuras - \includegraphics[scale=•]{•}
\usepackage{float} % ignorar regras de tipografia e inserir figura aonde queremos.
\usepackage[brazil]{babel} % Trocar Figure para Figura.
\usepackage{indentfirst}
\pagestyle{empty}


\begin{document}
\begin{figure}[H]
	\includegraphics[scale=0.9]{UnB_CiC_Logo.jpg}
\end{figure}
\noindent\rule{\textwidth}{0.4pt}
\begin{center}
	\textbf{{\Large Introdução à Ciência da Computação - 113913}} \newline \newline
	\textbf{{\large Prova 2} \\
	\vspace{9pt}
	{\large Questão B}} \\
	\noindent\rule{\textwidth}{0.4pt}
	\newline
\end{center}

\textbf{{\large Observações:}}
\begin{itemize}
	\item As provas também serão corrigidas por um \textbf{corretor automático}, portanto é necessário que as entradas e saídas do seu programa estejam conforme o padrão especificado em cada questão (exemplo de entrada e saída). Por exemplo, não use mensagens escritas durante o desenvolvimento do seu código como “Informe a primeira entrada”. Estas mensagens não são tratadas pelo corretor, portanto a correção irá resultar em resposta errada, mesmo que seu código esteja correto.
	\item Serão testadas várias entradas além das que foram dadas como exemplo, assim como as listas.
	\item Assim como as listas, as provas devem ser feitas na versão Python 3 ou superior.
	\item \textbf{Questão A valerá 40\% da nota da Prova 2 e a Questão B valerá 60\% da nota da Prova 2}.
	\item Leia com atenção e faça \textbf{exatamente} o que está sendo pedido.
\end{itemize}
\newpage % Questão A 
\begin{center}
\textbf{{\Large Questão B - Batalha de Pokémon}}
\end{center}
\vspace{5pt}
Pokémon Red é um jogo lançado em meados de 1996, e é bastante famoso até
hoje. Depois de onze anos, porém, decidiu que seria o momento de criar uma nova
feature para o jogo, o suporte à criação de torneios online. \newline \newline Mas para
isso, eles precisam de um programador python muito inteligente, por isso eles
chamaram você. Seu trabalho é definir quem é o vencedor de um torneio, fornecidos os resultados
das lutas. \newline \newline
\textbf{{\large Entrada}} \newline
A primeira linha da entrada consiste de um inteiro \textbf{P}, o número de pokémons
cadastrados no torneio atual. \newline
As próximas P linhas contém, cada uma, duas strings $\textbf{T}_P$ e \textbf{N}, o identificador de um pokémon no torneio e seu nome, respectivamente. \newline
As linhas seguintes descreverão cada uma das lutas do torneio e conterão, cada
uma, três strings $\textbf{T}_L$, o identificador desta luta, $\textbf{T}_{P_1}$ e $\textbf{T}_{P_2}$, os identificadores dos
lutadores envolvidos, e $\textbf{T}_V$, o vencedor. \newline
A última linha da entrada, por fim, possui uma string `\textbf{FINAL}', e o identificador da
luta final. \newline Note que não necessariamente a ordem de input será a ordem de execução das
lutas. \newline \newline
Considere sempre que dois pokémons diferentes nunca compartilharão o mesmo
nome, ou o mesmo identificador. \newline
Considere que pokémons nunca terão o mesmo identificador de uma luta. \newline
Perceba que $\textbf{T}_{P_1}$ e $\textbf{T}_{P_2}$ podem ser identificadores de lutadores ou de outras lutas. \newline
Neste caso, ele identifica o vencedor da luta identificada. \newline \newline
\textbf{{\large Saída}} \newline
Seu programa deve processar a entrada e imprimir na saída padrão uma única
linha contendo o nome do pokémon vencedor deste torneio.
\newline
\begin{table}[H]
\centering
\begin{tabular}{|l|l|}
\hline
\textbf{Exemplo de Entrada}                                                                                                                                                                                                                   & \textbf{Exemplo de Saída} \\ \hline
\begin{tabular}[c]{@{}l@{}}4\\ fire\_1 Charmander\\ water\_1 Squirtle\\ wind\_1 Pidgeot\\ mind\_1 Abra\\ fight\_5 fire\_1 water\_1 fire\_1\\ what wind\_1 mind\_1 mind\_1\\ final\_match what fight\_5 what\\ FINAL final\_match\end{tabular} & Abra                      \\ \hline
\begin{tabular}[c]{@{}l@{}}6\\ 10 NineTails\\ 0 Pikachu\\ 7 Sindaquill\\ 1 HitMonChan\\ 8 Articuno\\ 2 Suicune\\ 102 7 1 7\\ 100 0 8 0\\ 103 102 101 102\\ 104 100 103 100\\ 101 2 10 10\\ FINAL 104\end{tabular}                             & Pikachu                   \\ \hline
\end{tabular}
\end{table}
\flushright
\textbf{\Large Boa Prova!}
\end{document}