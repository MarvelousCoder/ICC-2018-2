\documentclass[a4paper, 12pt]{article}

\usepackage[top=2cm, bottom=2cm, left=2.5cm, right=2.5cm]{geometry}
\usepackage[utf8]{inputenc}
\usepackage{amsmath, amsfonts, amssymb}
\usepackage{graphicx} % inserir figuras - \includegraphics[scale=•]{•}
\usepackage{float} % ignorar regras de tipografia e inserir figura aonde queremos.
\usepackage[brazil]{babel} % Trocar Figure para Figura.
\usepackage{indentfirst}
\pagestyle{empty}


\begin{document}
\begin{figure}[H]
	\includegraphics[scale=0.9]{UnB_CiC_Logo.jpg}
\end{figure}
\noindent\rule{\textwidth}{0.4pt}
\begin{center}
	\textbf{{\Large Introdução à Ciência da Computação - 113913}} \newline \newline
	\textbf{{\large Prova 2} \\
	\vspace{9pt}
	{\large Questão B}} \\
	\noindent\rule{\textwidth}{0.4pt}
	\newline
\end{center}

\textbf{{\large Observações:}}
\begin{itemize}
	\item As provas serão corrigidas por um corretor automático, portanto é necessário que as entradas e saídas do seu programa estejam conforme o padrão especificado em cada questão (exemplo de entrada e saída).
	\item Por exemplo, não use mensagens escritas durante o desenvolvimento do seu código como “Informe a primeira entrada".
	\item Estas mensagens não são tratadas pelo corretor, portanto a correção irá resultar em resposta errada, mesmo que seu código esteja correto.
	\item Serão testadas várias entradas além das que foram dadas como exemplo, assim como as listas.
	\item Assim como as listas, as provas devem ser feitas na versão Python 3 ou superior.
	\item Cada questão (A e B) vale 50\% da nota da Prova 2.
	\item Leia com atenção e faça \textbf{exatamente} o que está sendo pedido.


\end{itemize}
\newpage % Questão B
\begin{center}
\textbf{{\Large Questão B - Detecção de Erros - ISBN-10}}
\end{center}

\vspace{5pt} 

Um dígito verificador é uma redundância utilizada para detecção de erros. É o equivalente decimal de uma paridade binária e é constituída por um ou mais dígitos calculados a partir dos outros dígitos da informação. Por exemplo, para a identificação de livros, em 1967 foi criado e oficializado como norma internacional em 1972, o ISBN - International Standard Book Number. Ele é um sistema que identifica numericamente os livros segundo o título, o autor, o país e a editora, individualizando-os inclusive por edição.
Os dois erros mais comuns no tratamento de um ISBN, quando escrito ou digitado, é um dígito errado ou a troca de dígitos adjacentes. O método de cálculo do dígito de verificação do ISBN garante que esses dois erros serão sempre detectados. No entanto, se o erro ocorre na editora e não é detectado, o livro será lançado com um ISBN inválido.
A fórmula para o cálculo do digito verificador do ISBN-13 é o que se segue abaixo:\newline
$$(10 – (x1 + 3x2 + x3 + 3x4 + ... +x11 + 3x12) mod 10) mod 10 = x13$$
\newline \newline
\textbf{{\large Entrada}} \newline
Diversos ISBN's, um por linha, onde cada algarismo é um coeficiênte da fórmula acima, começando por x1 e indo até x13. 
\newline
A letra X deve ser substituída pelo número 10. 
\newline
A entrada pode conter caracteres especiais e seu código deve ignorá-los.
\newline
O código deve continuar pedindo ISBN's enquanto não for inserido a palavra FIM.
\newline \newline
\textbf{{\large Saída}} \newline
Se o ISBN-13 estiver certo escreve OK, se estiver errado repete o ISBN-13 seguido da palavra ERRO.\newline
\newline \newline
\newline

\begin{table}[H]
\centering
\begin{tabular}{|l|l|}
\hline
\textbf{Entrada}                                                                                                              & \textbf{Saída}                                                      

\\ \hline
\begin{tabular}[c]{@{}l@{}} 853527782X\\ 978-8582602980\\ 8565848280\\ 978-8565848282\\ 8535286411\\ FIM\end{tabular}                                                                                                          & \begin{tabular}[c]{@{}l@{}}853527782X ERRO\\ OK\\ 8565848280 ERRO\\ OK\\ 8535286411 ERRO\end{tabular}
\\ \hline
\begin{tabular}[c]{@{}l@{}} 978-0671894412\\ 978/85658482*825\\ FIM\end{tabular}                                            & \begin{tabular}[c]{@{}l@{}}OK\\ 978/85658482*825 ERRO\end{tabular} 
\\ \hline

\end{tabular}
\caption{Questão B}
\end{table}

\flushright
\textbf{\Large Boa Prova!}
\end{document}