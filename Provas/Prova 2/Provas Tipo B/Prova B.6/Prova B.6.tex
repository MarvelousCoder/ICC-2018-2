\documentclass[a4paper, 12pt]{article}

\usepackage[top=2cm, bottom=2cm, left=2.5cm, right=2.5cm]{geometry}
\usepackage[utf8]{inputenc}
\usepackage{amsmath, amsfonts, amssymb}
\usepackage{graphicx} % inserir figuras - \includegraphics[scale=•]{•}
\usepackage{float} % ignorar regras de tipografia e inserir figura aonde queremos.
\usepackage[brazil]{babel} % Trocar Figure para Figura.
\usepackage{indentfirst}
\pagestyle{empty}


\begin{document}
\begin{figure}[H]
	\includegraphics[scale=0.9]{UnB_CiC_Logo.jpg}
\end{figure}
\noindent\rule{\textwidth}{0.4pt}
\begin{center}
	\textbf{{\Large Introdução à Ciência da Computação - 113913}} \newline \newline
	\textbf{{\large Prova 2} \\
	\vspace{9pt}
	{\large Questão B}} \\
	\noindent\rule{\textwidth}{0.4pt}
	\newline
\end{center}

\textbf{{\large Observações:}}
\begin{itemize}
	\item As provas também serão corrigidas por um \textbf{corretor automático}, portanto é necessário que as entradas e saídas do seu programa estejam conforme o padrão especificado em cada questão (exemplo de entrada e saída). Por exemplo, não use mensagens escritas durante o desenvolvimento do seu código como “Informe a primeira entrada”. Estas mensagens não são tratadas pelo corretor, portanto a correção irá resultar em resposta errada, mesmo que seu código esteja correto.
	\item Serão testadas várias entradas além das que foram dadas como exemplo, assim como as listas.
	\item Assim como as listas, as provas devem ser feitas na versão Python 3 ou superior.
	\item \textbf{Questão A valerá 40\% da nota da Prova 2 e a Questão B valerá 60\% da nota da Prova 2}.
	\item Leia com atenção e faça \textbf{exatamente} o que está sendo pedido.
\end{itemize}
\newpage % Questão A 
\begin{center}
\textbf{{\Large Questão B - Game Design}}
\end{center}
\vspace{5pt}
Um conceito muito conhecido no game design, na animação e na criação de robôs
é o \textit{Uncanny Valley}. Ele é a hipótese de que réplicas humanas que parecem quase
humanas, porém não completamente, despertam um sentimento de repulsa e
desconforto em seres humanos. \newline \newline
Depois do sucesso do seu jogo Elastiman, Roberto mais uma vez veio ao seu
contato. Ele está desenvolvendo um jogo realista, mas está se encontrando preso
no \textit{Uncanny Valley} por alguns diversos motivos. Seu trabalho será reconhecer
quais as \textit{features} que estão causando a perda de ``humanidade'' dos seus modelos
3D. \newline \newline
\textbf{{\large Entrada}} \newline
A primeira linha da entrada contém um inteiro \textbf{N}, o número de features a serem
descritas. \newline
As próximas \textbf{N} linhas contêm, cada uma, uma string $\textbf{F}_i$, o nome da \textit{feature} descrita. \newline
A linha seguinte, por sua vez, contém um inteiro \textbf{M}, o número de modelos a serem
analisados. \newline
As próximas \textbf{M} linhas contêm, cada uma, por fim, uma string de \textbf{N} caracteres, cada
um podendo ser \textbf{X}, caso o modelo atenda à \textit{feature}, ou \textbf{O} (a letra `O'), caso
contrário. Ou seja, se o \textbf{primeiro} caractere da string for um `X', esse modelo
atende a \textbf{primeira} \textit{feature}. \newline \newline
\textbf{{\large Saída}} \newline
Seu programa deve identificar dentre os modelos para análise, qual o que menos
atende às \textit{features} definidas (ou seja, o que mais contém `O's) e imprimir na saída
padrão quais as \textit{features} que ele \textbf{não} atende, na ordem de input, um em cada
linha. \newline
Considere que sempre haverá um único modelo com o número mínimo de \textbf{features}
atendidas.
\newline
\begin{table}[H]
\centering
\begin{tabular}{|l|l|}
\hline
\textbf{Exemplo de Entrada}                                                                                                              & \textbf{Exemplo de Saída}                                                                               \\ \hline
\begin{tabular}[c]{@{}l@{}}3\\ olhos vibrantes\\ texturas sólidas\\ cabelo com reflexão\\ 2\\ OXO\\ XOX\end{tabular}                     & \begin{tabular}[c]{@{}l@{}}olhos vibrantes\\ cabelo com reflexão\end{tabular}                           \\ \hline
\begin{tabular}[c]{@{}l@{}}2\\ ciclo de caminhada natural\\ animações de diálogo não-robóticas\\ 5\\ XX\\ OO\\ XO\\ OX\\ XX\end{tabular} & \begin{tabular}[c]{@{}l@{}}ciclo de caminhada natural\\ animações de diálogo não-robóticas\end{tabular} \\ \hline
\end{tabular}
\end{table}
\flushright
\textbf{\Large Boa Prova!}
\end{document}