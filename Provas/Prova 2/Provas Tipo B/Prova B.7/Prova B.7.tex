\documentclass[a4paper, 12pt]{article}

\usepackage[top=2cm, bottom=2cm, left=2.5cm, right=2.5cm]{geometry}
\usepackage[utf8]{inputenc}
\usepackage{amsmath, amsfonts, amssymb}
\usepackage{graphicx} % inserir figuras - \includegraphics[scale=•]{•}
\usepackage{float} % ignorar regras de tipografia e inserir figura aonde queremos.
\usepackage[brazil]{babel} % Trocar Figure para Figura.
\usepackage{indentfirst}
\pagestyle{empty}


\begin{document}
\begin{figure}[H]
	\includegraphics[scale=0.9]{UnB_CiC_Logo.jpg}
\end{figure}
\noindent\rule{\textwidth}{0.4pt}
\begin{center}
	\textbf{{\Large Introdução à Ciência da Computação - 113913}} \newline \newline
	\textbf{{\large Prova 2} \\
	\vspace{9pt}
	{\large Questão B}} \\
	\noindent\rule{\textwidth}{0.4pt}
	\newline
\end{center}

\textbf{{\large Observações:}}
\begin{itemize}
	\item As provas serão corrigidas por um corretor automático, portanto é necessário que as entradas e saídas do seu programa estejam conforme o padrão especificado em cada questão (exemplo de entrada e saída).
	\item Por exemplo, não use mensagens escritas durante o desenvolvimento do seu código como “Informe a primeira entrada".
	\item Estas mensagens não são tratadas pelo corretor, portanto a correção irá resultar em resposta errada, mesmo que seu código esteja correto.
	\item Serão testadas várias entradas além das que foram dadas como exemplo, assim como as listas.
	\item Assim como as listas, as provas devem ser feitas na versão Python 3 ou superior.
	\item Cada questão (A e B) vale 50\% da nota da Prova 2.
	\item Leia com atenção e faça \textbf{exatamente} o que está sendo pedido.


\end{itemize}
\newpage % Questão B
\begin{center}
\textbf{{\Large Questão B - Detecção de Erros - ISBN-10}}
\end{center}

\vspace{5pt} 

O Código Morse é um sistema de codificação de letras, números e sinais de pontuação que usa apenas dois símbolos, o ponto e o traço. Desenvolvido por Samuel Morse em 1835, ele é um código econômico, ou seja, foi projetado de tal forma que as letras mais frequentes na língua inglesa tivessem o código mais curto. Por exemplo, na língua inglesa o caracter que aparece com mais frequência em um texto é a letra “E”, representada por um ponto. Depois vem a letra “T” representada por um traço, em seguida o “A” representado por “.-” ponto-traço e assim por diante.

\renewcommand{\familydefault}{\ttdefault}
\begin{table}[H]
\centering
\begin{tabular}{|l|l|l|l|l|l|l|l|l|l|}
\hline
E & .   & H & .... & G & --.  & Q & --.-  & 7       & --...  \\ \hline
T & -   & L & .-.. & W & .--  & Z & --..  & 8       & ---..  \\ \hline
A & .-  & D & -..  & Y & -.-- & 1 & .---- & 9       & ----.  \\ \hline
O & --- & C & -.-. & B & -... & 2 & ..--- & 0       & -----  \\ \hline
I & ..  & U & ..-  & V & ...- & 3 & ...-- & ponto   & .-.-.- \\ \hline
N & -.  & M & --   & K & -.-  & 4 & ....- & vírgula & --..-- \\ \hline
S & ... & F & ..-. & X & -..- & 5 & ..... & ?       & ..--.. \\ \hline
R & .-. & P & .--. & J & .--- & 6 & -.... & -       & -....- \\ \hline
\end{tabular}
\end{table}

\renewcommand{\familydefault}{\rmdefault}
\noindent
\textbf{{\large Entrada}} \newline
Diversas palavras com símbolos alfanuméricos em Código Morse, um por linha, cada símbolo Morse separado por espaços em branco. 
\newline
A entrada de símbolos termina com a palavra “FIM” em Código Morse.
\newline
A palavra “FIM” não deve aparecer na saída.
\newline \newline
\textbf{{\large Saída}} \newline
As decodificações para o nosso alfabeto, uma por linha.\newline
\newline \newline
\newline

\begin{table}[H]
\centering
\begin{tabular}{|l|l|}
\hline
\textbf{Entrada}                                                                                                              & \textbf{Saída}                                                      

\\ \hline
\begin{tabular}[c]{@{}l@{}} \texttt{-.-. . --}\\ \texttt{.---- ----- -----}\\ \texttt{..-. .. --}\end{tabular}                                                                                                          & \begin{tabular}[c]{@{}l@{}}CEM\\ 100\end{tabular}
\\ \hline
\begin{tabular}[c]{@{}l@{}} \texttt{..-. .. --}\end{tabular}                                            & \begin{tabular}[c]{@{}l@{}}\end{tabular} 
\\ \hline

\end{tabular}
\caption{Questão B}
\end{table}

\flushright
\textbf{\Large Boa Prova!}
\end{document}