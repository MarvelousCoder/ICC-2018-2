\documentclass[a4paper, 12pt]{article}

\usepackage[top=2cm, bottom=2cm, left=2.5cm, right=2.5cm]{geometry}
\usepackage[utf8]{inputenc}
\usepackage{amsmath, amsfonts, amssymb}
\usepackage{graphicx} % inserir figuras - \includegraphics[scale=•]{•}
\usepackage{float} % ignorar regras de tipografia e inserir figura aonde queremos.
\usepackage[brazil]{babel} % Trocar Figure para Figura.
\usepackage{indentfirst}
\pagestyle{empty}


\begin{document}
\begin{figure}[H]
	\includegraphics[scale=0.9]{UnB_CiC_Logo.jpg}
\end{figure}
\noindent\rule{\textwidth}{0.4pt}
\begin{center}
	\textbf{{\Large Introdução à Ciência da Computação - 113913}} \newline \newline
	\textbf{{\large Prova 2} \\
	\vspace{9pt}
	{\large Questão B}} \\
	\noindent\rule{\textwidth}{0.4pt}
	\newline
\end{center}

\textbf{{\large Observações:}}
\begin{itemize}
	\item As provas também serão corrigidas por um \textbf{corretor automático}, portanto é necessário que as entradas e saídas do seu programa estejam conforme o padrão especificado em cada questão (exemplo de entrada e saída). Por exemplo, não use mensagens escritas durante o desenvolvimento do seu código como “Informe a primeira entrada”. Estas mensagens não são tratadas pelo corretor, portanto a correção irá resultar em resposta errada, mesmo que seu código esteja correto.
	\item Serão testadas várias entradas além das que foram dadas como exemplo, assim como as listas.
	\item Assim como as listas, as provas devem ser feitas na versão Python 3 ou superior.
	\item \textbf{Questão A valerá 40\% da nota da Prova 2 e a Questão B valerá 60\% da nota da Prova 2}.
	\item Leia com atenção e faça \textbf{exatamente} o que está sendo pedido.
\end{itemize}
\newpage % Questão A 
\begin{center}
\textbf{{\Large Questão B - Horários}}
\end{center}
\vspace{5pt}
Percival é dono de uma fábrica de sapatos para peixe, e deseja ter um controle
maior da entrada e saída dos seus funcionários. Como suas vendas têm
aumentado bastante nos últimos anos, ele tem tido um lucro maior e deseja fazer
maiores investimentos no setor de TI da sua empresa. \newline
Com um pouco de pesquisa, ele logo descobriu que você é o melhor
desenvolvedor python do Brasil e, portanto, a melhor opção para auxiliá-lo a
resolver seu objetivo. \newline
Seu trabalho é criar um software de consulta simples, de funcionários. \newline \newline
\textbf{{\large Entrada}} \newline
A primeira linha da entrada contém um inteiro \textbf{N}, o número de funcionários
cadastrados no sistema. \newline
As próximas \textbf{N} linhas contêm, cada uma, uma string sem espaços \textbf{N}, o nome do
funcionário, e duas strings $\textbf{H}_e$ e $\textbf{H}_s$, os horários de entrada e saída desse
funcionário, respectivamente. Considere que dois funcionários diferentes nunca
terão o mesmo nome. \newline
A última linha da entrada contém uma string $\textbf{H}_R$, o horário a ser consultado. \newline \newline
Considere que cada uma das strings de horário está no formato \textbf{hh:mm}, ou seja,
horas e minutos separados por um único semicolon (:), num dia de 24 horas. \newline
Considere também que ninguém irá atravessar a meia noite trabalhando. \newline \newline
\textbf{{\large Saída}} \newline
Seu programa deve imprimir uma única linha na saída padrão, contendo todos os
nomes dos funcionários que estariam trabalhando no horário requisitado,
separados por espaço, em ordem alfabética. \newline
\begin{table}[H]
\centering
\begin{tabular}{|l|l|}
\hline
\textbf{Exemplo de Entrada}                                                                                         & \textbf{Exemplo de Saída} \\ \hline
\begin{tabular}[c]{@{}l@{}}3\\ Fulano 03:00 08:00\\ Sicrano 05:00 09:00\\ Abelardo 07:00 11:00\\ 04:00\end{tabular} & Fulano                    \\ \hline
\begin{tabular}[c]{@{}l@{}}2\\ Yurick 09:30 19:30\\ André 07:00 22:00\\ 12:00\end{tabular}                          & André Yurick              \\ \hline
\end{tabular}
\end{table}
\flushright
\textbf{\Large Boa Prova!}
\end{document}