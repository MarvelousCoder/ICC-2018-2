\documentclass[a4paper, 12pt]{article}

\usepackage[top=2cm, bottom=2cm, left=2.5cm, right=2.5cm]{geometry}
\usepackage[utf8]{inputenc}
\usepackage{amsmath, amsfonts, amssymb}
\usepackage{graphicx} % inserir figuras - \includegraphics[scale=•]{•}
\usepackage{float} % ignorar regras de tipografia e inserir figura aonde queremos.
\usepackage[brazil]{babel} % Trocar Figure para Figura.
\usepackage{indentfirst}
\pagestyle{empty}


\begin{document}
\begin{figure}[H]
	\includegraphics[scale=0.9]{UnB_CiC_Logo.jpg}
\end{figure}
\noindent\rule{\textwidth}{0.4pt}
\begin{center}
	\textbf{{\Large Introdução à Ciência da Computação - 113913}} \newline \newline
	\textbf{{\large Prova 2} \\
	\vspace{9pt}
	{\large Questão B}} \\
	\noindent\rule{\textwidth}{0.4pt}
	\newline
\end{center}

\textbf{{\large Observações:}}
\begin{itemize}
	\item As provas também serão corrigidas por um \textbf{corretor automático}, portanto é necessário que as entradas e saídas do seu programa estejam conforme o padrão especificado em cada questão (exemplo de entrada e saída). Por exemplo, não use mensagens escritas durante o desenvolvimento do seu código como “Informe a primeira entrada”. Estas mensagens não são tratadas pelo corretor, portanto a correção irá resultar em resposta errada, mesmo que seu código esteja correto.
	\item Serão testadas várias entradas além das que foram dadas como exemplo, assim como as listas.
	\item Assim como as listas, as provas devem ser feitas na versão Python 3 ou superior.
	\item \textbf{Questão A valerá 40\% da nota da Prova 2 e a Questão B valerá 60\% da nota da Prova 2}.
	\item Leia com atenção e faça \textbf{exatamente} o que está sendo pedido.
\end{itemize}
\newpage % Questão A 
\begin{center}
\textbf{{\Large Questão B - Deambulação}}
\end{center}
\vspace{5pt}
Você estava andando pela sua já conhecida Cartésia, a cidade em que todas as
quadras são identificadas por coordenadas em um plano cartesiano, quando se
deparou com a sua velha amiga Cunegonde. \newline
Ela aparentava estar muito perdida, e se sentiu muito aliviada ao ver o melhor
programador que conhecia ali, bem na sua frente. Estava salva, enfim. \newline \newline
Cunegonde decidiu começar a fazer caminhadas regulares através de Cartésia, a
fim de começar a conhecer melhor a cidade. Mas o tiro foi pela culatra quando ela
começou a se perder durante suas próprias deambulações. Você, sendo o
excelente programador e altruísta que é, logo se disponibilizou para auxiliá-la a
identificar onde que seus roteiros de caminhada a levarão. \newline \newline
\textbf{{\large Entrada}} \newline
A primeira linha da entrada consiste de um inteiro \textbf{N}, o número de roteiros a ser
registrados no sistema. \newline
As próximas \textbf{N} linhas contém, cada uma, uma string sem espaços \textbf{S} e quatro
inteiros $\textbf{X}_o$, $\textbf{Y}_o$, $\textbf{D}_x$, $\textbf{D}_y$, o nome do roteiro e as coordenadas de onde Cunegonde irá
partir e o número de quadras que ela andou no sentido Leste e Norte,
respectivamente. Perceba que $\textbf{D}_x$, $\textbf{D}_y$ podem ser negativos caso ela tenha
andando no sentido Oeste ou Sul. \newline
A última linha da entrada contém, por fim, uma sequência de strings \textbf{R}, os
identificadores dos roteiros que Cunegonde quer seguir hoje. \newline
Considere que dois roteiros diferentes nunca terão o mesmo identificador.
\newline \newline
\textbf{{\large Saída}} \newline
Seu programa deve imprimir múltiplas linhas, uma para cada string \textbf{R} fornecida na
entrada, na ordem de input. Cada linha deve conter dois inteiros $\textbf{X}_F$ e $\textbf{Y}_F$, as
coordenadas da quadra final de Cunegonde após seguir o roteiro especificado. \newline
É sabido que em Cartésia as coordenadas crescem no sentido Leste e Norte, ou
seja, a posição (3, 0) fica mais à Leste do que (2, 0).
\newline
\begin{table}[H]
\centering
\begin{tabular}{|l|l|}
\hline
\textbf{Exemplo de Entrada}                                                                                                                                               & \textbf{Exemplo de Saída}                                  \\ \hline
\begin{tabular}[c]{@{}l@{}}3\\ caminhada\_da\_manhã 0 0 2 2\\ caminhada\_da\_tarde 2 2 3 3\\ rolezin 2 3 0 5\\ cainhada\_da\_manhã rolezin\end{tabular}                   & \begin{tabular}[c]{@{}l@{}}2 2\\ 2 8\end{tabular}          \\ \hline
\begin{tabular}[c]{@{}l@{}}4\\ papai\_noel 4 4 0 0\\ fogos\_de\_artificio 3 8 9 4\\ beijos 2 8 9 1\\ aboboras 0 0 0 0\\ aboboras fogos\_de\_artificio beijos\end{tabular} & \begin{tabular}[c]{@{}l@{}}0 0\\ 12 12\\ 11 9\end{tabular} \\ \hline
\end{tabular}
\end{table}
\flushright
\textbf{\Large Boa Prova!}
\end{document}