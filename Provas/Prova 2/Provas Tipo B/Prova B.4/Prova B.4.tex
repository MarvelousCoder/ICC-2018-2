\documentclass[a4paper, 12pt]{article}

\usepackage[top=2cm, bottom=2cm, left=2.5cm, right=2.5cm]{geometry}
\usepackage[utf8]{inputenc}
\usepackage{amsmath, amsfonts, amssymb}
\usepackage{graphicx} % inserir figuras - \includegraphics[scale=•]{•}
\usepackage{float} % ignorar regras de tipografia e inserir figura aonde queremos.
\usepackage[brazil]{babel} % Trocar Figure para Figura.
\usepackage{indentfirst}
\pagestyle{empty}


\begin{document}
\begin{figure}[H]
	\includegraphics[scale=0.9]{UnB_CiC_Logo.jpg}
\end{figure}
\noindent\rule{\textwidth}{0.4pt}
\begin{center}
	\textbf{{\Large Introdução à Ciência da Computação - 113913}} \newline \newline
	\textbf{{\large Prova 2} \\
	\vspace{9pt}
	{\large Questão B}} \\
	\noindent\rule{\textwidth}{0.4pt}
	\newline
\end{center}

\textbf{{\large Observações:}}
\begin{itemize}
	\item As provas também serão corrigidas por um \textbf{corretor automático}, portanto é necessário que as entradas e saídas do seu programa estejam conforme o padrão especificado em cada questão (exemplo de entrada e saída). \item Por exemplo, não use mensagens escritas durante o desenvolvimento do seu código como “Informe a primeira entrada”.
	\item Estas mensagens não são tratadas pelo corretor, portanto a correção irá resultar em resposta errada, mesmo que seu código esteja correto.
	\item Serão testadas várias entradas além das que foram dadas como exemplo, assim como as listas.
	\item Assim como as listas, as provas devem ser feitas na versão Python 3 ou superior.
	\item Cada questão (A e B) vale 50\% da nota da Prova 2.
	\item Leia com atenção e faça \textbf{exatamente} o que está sendo pedido.
\end{itemize}
\newpage % Questão A 
\begin{center}
\textbf{{\Large Questão B - Detecção de Erros – Paridade Ímpar}}
\end{center}
\vspace{5pt}
Toda informação armazenada ou transmitida está sujeita a erros e este fato não pode ser
simplesmente ignorado. Por isso a probabilidade de eles ocorrerem deve ser tratada. Uma forma
de tratar este problema é por meio de bits de paridade. Ou seja, para cada palavra em binário, é
acrescentado ou usa-se um ou mias bits da palavra para ser o bit de paridade. A paridade pode
ser par ou ímpar. Se for par, o número de bits 1 da palavra contando o bit de paridade deve ser
par, e ímpar caso contrário.
\newline \newline
\textbf{{\large Entrada}} \newline
Um número N em binário, com o bit menos significativo mais a direita, que é a quantidade de
palavras de sequência, seguido do seu valor em decimal.
\newline Sequência de palavras de 8, 16 ou 32 bits, sendo que o bit mais à direita é o bit de paridade
ímpar.
\newline \newline
\textbf{{\large Saída}} \newline
Um aviso dizendo se um erro foi detectado ou não na palavra transmitida. Se um erro for
detectado escrever “Erro detectado”, senão escrever “OK”.
\newline \newline

\newline
\begin{table}[H]
\centering
\begin{tabular}{|l|l|}
\hline
\textbf{Exemplo de Entrada}                                                                                                                                                                                                                                                                                                                                                                                                                                                                                                                                                                                                                                                                                                                                                                              & \textbf{Exemplo de Saída}                                                    \\ \hline
\begin{tabular}[c]{@{}l@{}}..- 1\\ .-.-.-.-.-..\end{tabular}                                                                                                                                                                                                                                                                                                                                                                                                                                                                                                                                                                                                                                                                                                                                             & OK                                                                           \\ \hline
\textasciicircum 0                                                                                                                                                                                                                                                                                                                                                                                                                                                                                                                                                                                                                                                                                                                                                                                       &                                                                              \\ \hline
\begin{tabular}[c]{@{}l@{}}\# 1\\ \#+\#+\#+\#+\end{tabular}                                                                                                                                                                                                                                                                                                                                                                                                                                                                                                                                                                                                                                                                                                                                              & Erro detectado                                                               \\ \hline
\begin{tabular}[c]{@{}l@{}}\textless{}\textless 3\\ \textless{}\textgreater{}\textless{}\textgreater{}\textless{}\textgreater{}\textless{}\textgreater{}\textless{}\textgreater{}\textless{}\textgreater{}\textless{}\textgreater{}\textless{}\textgreater{}\textless{}\textgreater{}\textless{}\textgreater{}\textless{}\textgreater{}\textless{}\textgreater{}\textless{}\textgreater{}\textless{}\textgreater{}\textless{}\textgreater{}\textless{}\textgreater\\ \textgreater{}\textless{}\textgreater{}\textless{}\textgreater{}\textless{}\textgreater{}\textless{}\textgreater{}\textless{}\textgreater{}\textless{}\textgreater{}\textless{}\textgreater{}\textless\\ \textgreater{}\textgreater{}\textgreater{}\textgreater{}\textgreater{}\textgreater{}\textgreater{}\textless{}\end{tabular} & \begin{tabular}[c]{@{}l@{}}Erro detectado\\ Erro detectado\\ OK\end{tabular} \\ \hline

\end{tabular}
\end{table}
\flushright
\textbf{\Large Boa Prova!}
\end{document}