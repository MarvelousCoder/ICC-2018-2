\documentclass[a4paper, 12pt]{article}

\usepackage[top=2cm, bottom=2cm, left=2.5cm, right=2.5cm]{geometry}
\usepackage[utf8]{inputenc}
\usepackage{amsmath, amsfonts, amssymb}
\usepackage{graphicx} % inserir figuras - \includegraphics[scale=•]{•}
\usepackage{float} % ignorar regras de tipografia e inserir figura aonde queremos.
\usepackage[brazil]{babel} % Trocar Figure para Figura.
\usepackage{indentfirst}
\pagestyle{empty}


\begin{document}
\begin{figure}[H]
	\includegraphics[scale=0.9]{UnB_CiC_Logo.jpg}
\end{figure}
\noindent\rule{\textwidth}{0.4pt}
\begin{center}
	\textbf{{\Large Introdução à Ciência da Computação - 113913}} \newline \newline
	\textbf{{\large Prova 2} \\
	\vspace{9pt}
	{\large Questão B}} \\
	\noindent\rule{\textwidth}{0.4pt}
	\newline
\end{center}

\textbf{{\large Observações:}}
\begin{itemize}
	\item As provas também serão corrigidas por um \textbf{corretor automático}, portanto é necessário que as entradas e saídas do seu programa estejam conforme o padrão especificado em cada questão (exemplo de entrada e saída). Por exemplo, não use mensagens escritas durante o desenvolvimento do seu código como “Informe a primeira entrada”. Estas mensagens não são tratadas pelo corretor, portanto a correção irá resultar em resposta errada, mesmo que seu código esteja correto.
	\item Serão testadas várias entradas além das que foram dadas como exemplo, assim como as listas.
	\item Assim como as listas, as provas devem ser feitas na versão Python 3 ou superior.
	\item \textbf{Questão A valerá 40\% da nota da Prova 2 e a Questão B valerá 60\% da nota da Prova 2}.
	\item Leia com atenção e faça \textbf{exatamente} o que está sendo pedido.
\end{itemize}
\newpage % Questão A 
\begin{center}
\textbf{{\Large Questão B - Javascript}}
\end{center}
\vspace{5pt}
Florêncio é um programador Ruby que, depois de muito esforço, conseguiu se
acostumar à sintaxe CamelCase do Java. Só que agora Florêncio deseja aprender a criar frameworks Javascript (meio que está na moda), e está tendo dificuldades em se acostumar a nomear suas
bibliotecas usando o padrão kebab-case. \newline \newline
Lembrando que você o ajudou da última vez, Florêncio pede sua ajuda para
escrever um conversor de snake\_case e CamelCase para kebab-case. \newline \newline
\textbf{{\large Entrada}} \newline
A primeira e única linha da entrada consiste de uma palavra ou frase em
snake\_case ou CamelCase ou um misto das duas.
No snake\_case, as palavras estão todas em minúsculo, separadas por
\textit{underscores} (\_). \newline
No CamelCase, as palavras estão todas em minúsculo, exceto a primeira letra de
cada palavra, que está capitalizada. \newline \newline
\textbf{{\large Saída}} \newline
Seu programa deve imprimir uma única linha, contendo a entrada convertida para
kebab-case, ou seja, todas as palavras em minúsculo, separados por hífens (-).
\newline
\begin{table}[H]
\centering
\begin{tabular}{|l|l|}
\hline
\textbf{Exemplo de Entrada}       & \textbf{Exemplo de Saída}         \\ \hline
snake\_case\_CamelCase            & snake-case-camel-case             \\ \hline
create\_underscoredBookCock\_tail & create-underscored-book-cock-tail \\ \hline
\end{tabular}
\end{table}
\flushright
\textbf{\Large Boa Prova!}
\end{document}