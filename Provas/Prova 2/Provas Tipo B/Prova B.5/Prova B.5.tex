\documentclass[a4paper, 12pt]{article}

\usepackage[top=2cm, bottom=2cm, left=2.5cm, right=2.5cm]{geometry}
\usepackage[utf8]{inputenc}
\usepackage{amsmath, amsfonts, amssymb}
\usepackage{graphicx} % inserir figuras - \includegraphics[scale=•]{•}
\usepackage{float} % ignorar regras de tipografia e inserir figura aonde queremos.
\usepackage[brazil]{babel} % Trocar Figure para Figura.
\usepackage{indentfirst}
\pagestyle{empty}


\begin{document}
\begin{figure}[H]
	\includegraphics[scale=0.9]{UnB_CiC_Logo.jpg}
\end{figure}
\noindent\rule{\textwidth}{0.4pt}
\begin{center}
	\textbf{{\Large Introdução à Ciência da Computação - 113913}} \newline \newline
	\textbf{{\large Prova 2} \\
	\vspace{9pt}
	{\large Questão B}} \\
	\noindent\rule{\textwidth}{0.4pt}
	\newline
\end{center}

\textbf{{\large Observações:}}
\begin{itemize}
	\item As provas também serão corrigidas por um \textbf{corretor automático}, portanto é necessário que as entradas e saídas do seu programa estejam conforme o padrão especificado em cada questão (exemplo de entrada e saída). Por exemplo, não use mensagens escritas durante o desenvolvimento do seu código como “Informe a primeira entrada”. Estas mensagens não são tratadas pelo corretor, portanto a correção irá resultar em resposta errada, mesmo que seu código esteja correto.
	\item Serão testadas várias entradas além das que foram dadas como exemplo, assim como as listas.
	\item Assim como as listas, as provas devem ser feitas na versão Python 3 ou superior.
	\item \textbf{Questão A valerá 40\% da nota da Prova 2 e a Questão B valerá 60\% da nota da Prova 2}.
	\item Leia com atenção e faça \textbf{exatamente} o que está sendo pedido.
\end{itemize}
\newpage % Questão A 
\begin{center}
\textbf{{\Large Questão B - Estado}}
\end{center}
\vspace{5pt}
Estamos vivendo um período de crise política, e, cada dia mais, não é segredo
para ninguém que a maioria dos nossos políticos está envolvido em algum
esquema de desvio de dinheiro público. \newline
Como o cidadão consciente e ativista que é, você decidiu usar suas habilidades
como programador para o bem e criar um software que analisa as contas públicas
e identifica potenciais corruptos usando inteligência artificial. \newline \newline
Porém, em uma das últimas semanas de trabalho intenso e de afinco, você se
deparou com um problema. Você precisa de um programa que encontre políticos
pelas denúncias a ele associadas. \newline \newline
\textbf{{\large Entrada}} \newline
A primeira linha da entrada contém um inteiro \textbf{N}, o número de políticos registrados
no sistema. \newline
As próximas \textbf{N} linhas contêm, cada uma, uma string sem espaços \textbf{P}, o nome do
político, e uma sequência de strings $\textbf{D}_P$, as denúncias a ele associadas. Considere
que dois políticos diferentes nunca terão o mesmo nome. \newline
A última linha da entrada contém uma sequência de strings $\textbf{D}_R$, as denúncias a
serem pesquisadas. \newline \newline
\textbf{{\large Saída}} \newline
Seu programa deve imprimir uma única linha na saída padrão, o nome dos
políticos que têm envolvimento em pelo menos um dos esquemas de corrupção
citados, em ordem alfabética. \newline
\begin{table}[H]
\centering
\begin{tabular}{|l|l|}
\hline
\textbf{Exemplo de Entrada}                                                                                                                                     & \textbf{Exemplo de Saída}       \\ \hline
\begin{tabular}[c]{@{}l@{}}3\\ Jobernardo Furnas Petrobrás Odebrecht\\ MariaDeSouza Odebrecht JBS\\ JulioRoberto Furnas Petrobrás\\ Odebrecht\end{tabular}      & Jobernardo MariaDeSouza         \\ \hline
\begin{tabular}[c]{@{}l@{}}4\\ JuliaAlala Volkswagen Fiat\\ MulioRobaldo Fiat Uno\\ Nadindo Volkswagen Uno\\ KatioJumbo Fiat Fiat\\ Uno Volkswagen\end{tabular} & JuliaAlala MulioRobaldo Nadindo \\ \hline
\end{tabular}
\end{table}
\flushright
\textbf{\Large Boa Prova!}
\end{document}