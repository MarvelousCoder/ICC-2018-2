\documentclass[a4paper, 12pt]{article}

\usepackage[top=2cm, bottom=2cm, left=2.5cm, right=2.5cm]{geometry}
\usepackage[utf8]{inputenc}
\usepackage[utf8x]{inputenc}
\usepackage{amsmath, amsfonts, amssymb}
\usepackage{graphicx} % inserir figuras - \includegraphics[scale=•]{•}
\usepackage{float} % ignorar regras de tipografia e inserir figura aonde queremos.
\usepackage[brazil]{babel} % Trocar Figure para Figura.
\usepackage{indentfirst}
\pagestyle{empty}


\begin{document}
\begin{figure}[H]
	\includegraphics[scale=0.9]{UnB_CiC_Logo.jpg}
\end{figure}
\noindent\rule{\textwidth}{0.4pt}
\begin{center}
	\textbf{{\Large Introdução à Ciência da Computação - 113913}} \newline \newline
	\textbf{{\large Prova 2} \\
	\vspace{9pt}
	{\large Questão B}} \\
	\noindent\rule{\textwidth}{0.4pt}
	\newline
\end{center}

\textbf{{\large Observações:}}
\begin{itemize}
	\item As provas também serão corrigidas por um \textbf{corretor automático}, portanto é necessário que as entradas e saídas do seu programa estejam conforme o padrão especificado em cada questão (exemplo de entrada e saída). Por exemplo, não use mensagens escritas durante o desenvolvimento do seu código como “Informe a primeira entrada”. Estas mensagens não são tratadas pelo corretor, portanto a correção irá resultar em resposta errada, mesmo que seu código esteja correto.
	\item Serão testadas várias entradas além das que foram dadas como exemplo, assim como as listas.
	\item Assim como as listas, as provas devem ser feitas na versão Python 3 ou superior.
	\item \textbf{Cada questão (A e B) vale 50\% da nota da prova 2}.
	\item Leia com atenção e faça \textbf{exatamente} o que está sendo pedido.
\end{itemize}
\newpage % Questão A 
\begin{center}
\textbf{{\Large Questão B - Código Morse}}
\end{center}
\vspace{5pt}

O Código Morse é um sistema de codificação de letras, números e sinais de pontuação que usa apenas dois símbolos, o ponto e o traço. Desenvolvido por Samuel Morse em 1835, ele é um código econômico, ou seja, foi projetado de tal forma que as letras mais frequentes na língua inglesa tivessem o código mais curto. Por exemplo, na língua inglesa o caracter que aparece com mais frequência em um texto é a letra “E”, representada por um ponto. Depois vem a letra “T” representada por um traço, em seguida o “A” representado por “.-“ ponto-traço e assim por diante.
\begin{table}[htb]
\centering
\begin{tabular}{|l|l|l|l|l|l|l|l|l|l|}
\hline
E & .   & H & ...  & G & --.  & Q & --.-  & 7       & --...  \\ \hline
T & -   & L & .-.. & W & .--  & Z & --..  & 8       & ---..  \\ \hline
A & .-  & D & -..  & Y & -.-- & 1 & .---- & 9       & ----.  \\ \hline
O & --- & C & -.-. & B & -... & 2 & ..--- & 0       & -----  \\ \hline
I & ..  & U & ..-  & V & ...- & 3 & ...-- & ponto   & .-.-.- \\ \hline
N & -.  & M & --   & K & -.-  & 4 & ....- & virgula & --..-- \\ \hline
S & ... & F & ..-. & X & -..- & 5 & ..... & ?       & ..--.. \\ \hline
R & .-. & P & .--. & J & .--- & 6 & -.... & -       & -....- \\ \hline
\end{tabular}
\end{table}
\newline \newline
\textbf{{\large Entrada}} \newline

Diversas palavras com símbolos da tablea acima, na forma de texto, um por linha. A lista de símbolos termina com a palavra "FIM". A palavra "FIM"  não faz parte da lista de palavras.
\newline \newline
\textbf{{\large Saída}} \newline

As palavras em Código Morse, sendo que os caracteres em Morse devem estar separados por um espaço em branco.
\\
\newline \newline
\textbf{{\large Exemplo}} \newline

\begin{table}[htb]
\centering
\begin{tabular}{|l|l|}
\hline
Entrada                                                                           & Saída                                                                                                                                                      \\ \hline
FIM                                                                               &                                                                                                                                                            \\ \hline
\begin{tabular}[c]{@{}l@{}}SOS\\ FIM\end{tabular}                                 & ... --- ...                                                                                                                                                \\ \hline
\begin{tabular}[c]{@{}l@{}}CEM\\ 100\\ FIM\end{tabular}                           & \begin{tabular}[c]{@{}l@{}}-.-. . --\\ .---- ----- -----\end{tabular}                                                                                      \\ \hline
\begin{tabular}[c]{@{}l@{}}BANANA\\ ABACAXI\\ GOIABA\\ GP09-02\\ FIM\end{tabular} & \begin{tabular}[c]{@{}l@{}}-... .- -. .- -. .-\\ .- -... .- -.-. .- -..- ..\\ --. --- .. .- -... .-\\ --. .--. ----- ----. -....- ----- ..---\end{tabular} \\ \hline
\end{tabular}
\end{table}


\flushright
\textbf{\Large Boa Prova!}
\end{document}